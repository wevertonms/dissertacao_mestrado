% !TeX root = ./dissertacao.tex

% ---------------------------------------------------------------------------
% ---------------------------------------------------------------------------
% Modelo LaTex para preparação do documento final de Dissertação de Mestrado
% O modelo está em conformidade com ABNT NBR 14724:2011:
% Programa de Pós-Graduação em Informática
% Universidade Federal de Alagoas
% Versão: v0.9
% ---------------------------------------------------------------------------
% ---------------------------------------------------------------------------
\PassOptionsToPackage{inline}{enumitem}

\documentclass[
	% -- opções da classe memoir --
	12pt,					% tamanho da fonte
	openright,				% capítulos começam em páginas ímpares (insere página vazia caso preciso)
	oneside,				% para impressão em verso e anverso. Oposto a oneside
	a4paper,				% tamanho do papel.
	% -- opções da classe abntex2 --
	chapter=TITLE,			% títulos de capítulos convertidos em letras maiúsculas
	%section=TITLE,			% títulos de seções convertidos em letras maiúsculas
	%subsection=TITLE,		% títulos de subseções convertidos em letras maiúsculas
	%subsubsection=TITLE,	% títulos de subsubseções convertidos em letras maiúsculas
	% -- opções do pacote babel --
	english,				% idioma adicional para hifenização
	brazil					% o último idioma é o principal do documento
]{abntex2}

% --------------------
% Pacotes OBRIGATÓRIOS
% --------------------
\usepackage[T1]{fontenc}		% Seleção de códigos de fonte.
\usepackage[utf8]{inputenc}		% Codificação do documento (conversão automática dos acentos)
\usepackage{inconsolata}
\usepackage{lastpage}			% Usado pela Ficha catalográfica
\usepackage{indentfirst}		% Indenta o primeiro parágrafo de cada seção.
\usepackage{color}				% Controle das cores
\usepackage{graphicx,graphicx}	% Inclusão de gráficos
% \usepackage{epsfig,subfig}	% Inclusão de figuras
\usepackage{microtype} 			% Melhorias de justificação
% ---------------------

% ---------------------
% Pacotes ADICIONAIS
% ---------------------
\usepackage[newfloat=true]{minted}      % don't need to import `listing` or `listings`
\usepackage{amsmath,amssymb,mathrsfs}	% Comandos matemáticos avançados
\usepackage{setspace}  					% Para permitir espaçamento simples, 1 1/2 e duplo
\usepackage{verbatim}					% Para poder usar o ambiente "comment"
\usepackage{tabularx} 					% Para poder ter tabelas com colunas de largura auto-ajustável
\usepackage{afterpage} 					% Para executar um comando depois do fim da página corrente
\usepackage{url} 			            % Para formatar URLs (endereços da Web)
\usepackage{svg}
% \counterwithin{figure}{chapter}         % Para numerar figuras por capítulo
\usepackage{enumerate}
\usepackage{units}
\usepackage{caption}
\usepackage{subcaption}
\usepackage{float}
\usemintedstyle{emacs}
\newminted{json}{gobble=2,frame=single,fontsize=\tiny}
\newminted{python}{frame=single,fontsize=\tiny,linenos}
\usepackage{pdfpages} % Comando includepdf
\usepackage{lscape}   % landscape
\usepackage{silence} % Silenciar erros e warnings
\WarningsOff
% ---------------------
%
% ---------------------
% Pacotes de CITAÇÕES
% ---------------------
\usepackage[brazilian,hyperpageref]{backref}  % Paginas com as citações na bibliografia
\usepackage[alf]{abntex2cite}				  % Citações padrão ABNT (alfa)
%\usepackage[num]{abntex2cite}				  % Citações padrão ABNT (numéricas)
% ---------------------

% Definição de diretório de imagens
\graphicspath{{imagens/}}

% Configurações de CITAÇÕES para abntex2
\include{extras/conf_citacoes}

% Inclusão de dados para CAPA e FOLHA DE ROSTO (título, autor, orientador, etc.)
% ---
% Informações de dados para CAPA e FOLHA DE ROSTO
% ---
\titulo{FRAMEWORK INTEGRADO PARA ANÁLISE DE FADIGA EM DUTOS SUBMARINOS EM VÃO-LIVRE}
\autor{Weverton Marques da Silva}
\local{Maceió-AL}
\data{Dezembro de 2020}
\orientador{Adeildo Soares Ramos Júnior}
\coorientador{Eduardo Setton S. da Silveira}
\instituicao{%
  Universidade Federal de Alagoas --- UFAL
  \par
  Centro de Tecnologia
  \par
  Programa de Pós-Graduação em Engenharia Civil
}
\tipotrabalho{Dissertação (Mestrado)}
% O preambulo deve conter o tipo do trabalho, o objetivo,
% o nome da instituição e a área de concentração
\preambulo{Dissertação apresentada como requisito parcial para obtenção do grau de Mestre pelo Programa de Pós-Graduação em Engenharia Civil do Centro de Tecnologia da Universidade Federal de Alagoas.}
% ---


% Inclui Configurações de aparência do PDF Final
\include{extras/conf_pdf}

% Inclui configurações da folha de rosto
\include{extras/conf_folha_de_rosto}

% Configurações da fonte do texto
\include{extras/conf_font}

% O tamanho da indentação do parágrafo é dado por:
\setlength{\parindent}{1.3cm}

% Controle do espaçamento entre um parágrafo e outro:
\setlength{\parskip}{0.2cm}  % tente também \onelineskip

% Espaço entre o título do capítulo e o texto
\setlength{\afterchapskip}{0.8cm}

% ---------------------
% Compila o índice
% ---------------------
\makeindex
% ---------------------

%%%%%%%%%%%%%%%%%%%%%%%%%%%
%%  INICIO DO DOCUMENTO  %%
%%%%%%%%%%%%%%%%%%%%%%%%%%%
\begin{document}

% Elementos textuais com numeração arábica
\pagenumbering{arabic}

% Retira espaço extra obsoleto entre as frases.
\frenchspacing

% ----------------------------------------------------------
% ELEMENTOS PRÉ-TEXTUAIS (Capa, Resumo, Abstract, etc.)
% ----------------------------------------------------------
\pretextual

% Capa
% ---
% Impressão da Capa
% ---
\begin{capa}
	\center
	UNIVERSIDADE FEDERAL DE ALAGOAS\\
	CENTRO DE TECNOLOGIA\\
	PROGRAMA DE PÓS GRADUAÇÃO EM ENGENHARIA CIVIL

	\vfill
	\MakeUppercase{\imprimirautor}

    \vfill
    \textbf{\imprimirtitulo}
    \vfill

    \vfill
    \imprimirlocal
    \par
    \imprimirdata

    \vspace*{1cm}
\end{capa}
% ---

% Folha de rosto (o * indica que haverá a ficha bibliográfica)
\imprimirfolhaderosto\

% Imprimir Ficha Catalográfica
% \include{pretextual/catalografica}

% Inserir Folha de Aprovação
% \include{pretextual/assinaturas}

% Dedicatória
% ---
% Dedicatória
% ---
\begin{dedicatoria}
   \vspace*{\fill}
   \centering
   \noindent
   \textit{Ao meu falecido pai, Joel.} \vspace*{\fill}
\end{dedicatoria}
% ---

% Agradecimentos
% ---
% Agradecimentos
% ---
\begin{agradecimentos}

Aos meus familiares por todo apoio. Especialmente, ao meus irmão, Wagner, e a minha noiva, Jhulia, que estiveram sempre ao meu lado e me ajudaram a encontrar a disposição para seguir em frente.

Agradeço ao meu orientador e coorientador pelos direcionamentos e por acreditaram na importância dos frutos desse trabalho.

Ao Laboratório de Computação Científica e Visualização pela minha participação no projeto IntegriSpan. Especialmente aos meus amigos Emerson, Josué, Jéssica e Renato. Sem a ajuda e parceria inestimável deles este trabalho não seria possível.

Aos professores do Programa de Pós-Graduação em Engenharia Civil que, com os conhecimentos transmitidos nas disciplinas, me ajudaram a elaborar esse trabalho.

\end{agradecimentos}
%% ---

% Epígrafe
% ---
% Epígrafe
% ---
\begin{epigrafe}
    \vspace*{\fill}
	\begin{flushright}
		\textit{``Seja curioso. Leia muito. Experimente coisas novas.\\ Eu acho que muito do que as pessoas chamam de inteligência\\apenas se resume a curiosidade.''\\
		          (Aaron Swartz)}
	\end{flushright}
\end{epigrafe}
% ---

% Resumo e Abstract
% ---
% RESUMOS
% ---

% RESUMO em português
\setlength{\absparsep}{18pt} % ajusta o espaçamento dos parágrafos do resumo
\begin{resumo}

    Com as descobertas de novos campos de petróleo em águas profundas, e distantes da costa, surgiu a necessidade de sistemas de transporte submarinos utilizando dutos rígidos cada vez mais extensos, e consequentemente com maior propensão à ocorrência de vãos-livres induzidos por irregularidades do piso marinho.
    A presença desses vãos traz a possibilidade de haver vibrações induzidas por vórtices (VIV) desprendidos pela passagem da corrente.
    Agora, anos décadas após instalação dessa estruturas, é chegada a hora de revisitar as previsões para vida útil e avaliar a necessidade de intervenção no sentido de extender a vída útil das mesmas.
    O início dessas e outras análises se baseiam no comportamento \textit{in-loco} do duto.
    O Método dos Elementos Finitos é amplamente empregado na modelagem desse comportamento.
    No entanto, essa não é uma tarefa trivial, uma vez que depende da utilização de vários softwares para etapas intermediárias.
    Além disso, essas análises envolvem manipulação de grandes quantidades de dados relacionados à batimetria do fundo marinho e propriedades dos materiais, geometria do duto, correnteza, entre outros.
    Dessa forma, todo esse processo costuma ser demorado e passível de erros.
    Concomitantemente, a indústria de óleo e gás tem passado por uma transformação digital nos anos recentes. Este cenário trouxe novas oportunidades para o desenvolvimento de novas ferramentas para dia-a-dia do profissional responsável por estas análises.
    Este trabalho apresenta o desenvolvimento de uma ferramenta para agilizar o fluxo de análise de fadiga em dutos submarinos sujeitos a vãos livres.
    

 \textbf{Palavras-chaves}: Dutos Submarinos; Modelagem Numérica; Modelagem Computacional.
\end{resumo}

% ABSTRACT in english
\begin{resumo}[Abstract]
    \begin{otherlanguage*}{english}
    
    With the discovery of new deep-sea oil fields far from the coast, has arisen the need for submarine transportation systems using increasingly large rigid pipelines, and consequently with greater propensity for spans induced by seabed irregularities.
    The presence of these spans brings the possibility of vortex-induced vibrations (VIV) produced by the current flow.
    Thus, an assessment is required to determine whether corrective actions are required to prevent damage to them.
    Now, decades after the installation of these structures, the time has come to analyze the predictions for their lifespan and to assess the need for intervention to extend their lifespan.
    The start of these and other analyzes is based on in-situ pipeline behavior.
    The Finite Element Method is widely applied in modeling this behavior.
    However, this is not a trivial task as it depends on using various software for intermediate steps.
    Moreover, these analyses involve manipulation of large amounts of data related to seabed bathymetry and material properties, duct geometry, current, among others.
    This whole process is often time-consuming and error prone.
    In parallel, the oil and gas industry has undergone a digital transformation in recent years.
    This scenario brought new opportunities for the development of new tools the daily of the person responsible for these analyzes.
    This work presents the development of a tool to speed up the flow of fatigue analysis in subsea pipelines subjected to free spans.

    \vspace{\onelineskip}

    \noindent
    \textbf{Keywords}: Subsea Pipeline; Numerical Modeling; Computacional Modeling.
    \end{otherlanguage*}
\end{resumo}

% Lista de ilustrações
\pdfbookmark[0]{\listfigurename}{lof}
\listoffigures*
\cleardoublepage

% Lista de tabelas
\pdfbookmark[0]{\listtablename}{lot}
\listoftables*
\cleardoublepage

% Lista de abreviaturas e siglas
\begin{siglas}
  \item[CFD] \textit{Computer Fluid Dynamics}
  \item[CLI] \textit{Command Line Interface}
  \item[MEF] Método dos Elementos Finitos
  \item[VBA] \textit{Visual Basic for Applications}
  \item[VIV] Vibração Induzida por Vórtice
  \item[UML] \textit{Unified Modeling Language}
\end{siglas}

% Lista de símbolos
%\begin{simbolos}
%  \item[$ \Gamma $] Letra grega Gama
%  \item[$ \Lambda $] Lambda
%  \item[$ \zeta $] Letra grega minúscula zeta
%  \item[$ \in $] Pertence
%\end{simbolos}

% Inserir o SUMÁRIO
\pdfbookmark[0]{\contentsname}{toc}
\settocdepth{section}
\tableofcontents*
\cleardoublepage

% ----------------------------------------------------------
% ELEMENTOS TEXTUAIS (Capítulos)
% ----------------------------------------------------------
\textual

% Ajusta o header para conter apenas o número da página
\pagestyle{simple}

% Atalhos para escritas comuns
\def\fatfree{FatFree}
\def\dnvf105{DNVGL-RP-F105}
\def\integrispan{IntegriSpan}
% ----------------------------------------------------------
% TEXTO DA DISSERTAÇÃO
% ----------------------------------------------------------
\chapter{Introdução}


Ao longo das últimas décadas, à medida que novos campos de petróleo e gás foram descobertos em águas profundas e distantes da costa, surgiu a necessidade de utilização de sistemas de coleta e exportação submarinos utilizando dutos rígidos cada vez mais extensos.
Com uma maior extensão, aumentou-se a ocorrência de seções de duto que não ficam apoiadas sob o leito marinho devido as irregularidades do solo. Essas seções dos dutos são denominamos trechos em vão-livre. Estas irregularidades podem ser preexistentes durante a instalação ou devido a subsequentes movimentos horizontais de \textit{scouring}\footnote{retirada de solo que suporta o duto devido às intensas correntes de fundo.} de dutos durante a operação.

A presença de trechos dos dutos em vão-livre exige uma avaliação para determinação da necessidade de ações corretivas para evitar danos aos mesmos.
Ainda na fase de projeto, uma avaliação do perfil do fundo do mar ao longo da rota proposta pode ser realizada para identificar se é esperado que haja trechos do duto em vão-livre.
Na existência de tais trechos, será necessária uma análise numérica que forneça previsões dos números e tamanhos dos vãos esperados, que são indicadores da necessidade de possíveis alterações na rota ou ações corretivas.

Devido aos elevados custos ambientais, financeiros, e à imagem da empresa, associados aos acidentes, o transporte seguro de hidrocarbonetos e outros fluidos nos oleodutos é uma das principais prioridades da indústria de petróleo e gás.
A vibração livre é uma grande preocupação na análise de fadiga de componentes de dutos submarinos, incluindo dutos em vãos-livres~\cite{Gamino2013}.

Sendo assim, o comportamento estático e dinâmico do duto deve ser investigado para garantir a segurança, combatendo o dano estrutural por fadiga, mantendo-o em um estado aceitavelmente seguro.
Se as condições necessárias à segurança não puderem ser garantidas, as ações corretivas na forma de mudança de rota, correção de vãos, supressão do VIV e similares são usadas para garantir que os critérios de projeto relativos aos níveis de tensão e possíveis danos por fadiga devido ao VIV não sejam excedidos.
Para a definição mais assertivas e precisas a modelagem estrutural deve ser utilizada, e o Método dos Elementos Finitos (MEF) é amplamente usado nessa tarefa.
A configuração de dutos no fundo do mar depende das características topográficas do leito marinho, características do solo, tensão residual de lançamento, rigidez do duto e seu peso submerso.

Para que as condições de contorno e características do problema simulado reproduzam comportamento real, é necessário modelar desde a etapa de instalação até a operação do duto, assim como considerar efeito de carregamentos dos diferentes valores de pressões internas e externas nas respectivas etapas.
Modelar a instalação de dutos em um software de elementos finitos para uso geral pode ser um trabalho demorado e tedioso, principalmente devido a grandes quantidades de dados da batimetria.
Na maioria das vezes, é necessário até o uso de \textit{script} na definição do perfil do leito marinho, para poder simular o processo de assentamento~\cite{VandenAbeele2013}.

Nesse cenário, são essenciais ferramentas que possam auxiliar não só no processamento (como os softwares de simulação), mas também o pré e pós-processamento de dados e, até mesmo, na automação de procedimentos.
Uma ferramenta com essas características traz ganhos significativos para a produtividade e reduz a possibilidade de ocorrência de erros humanos.
Além de aumento de produtividade a redução da falha de erro humano, tem as questões ligadas a difícil mobilização e custos dessas operações. Que, caso sejam feitas de forma inadequada, podem até amplificar o problema.
Adicionalmente, uma ferramenta que integre softwares de uso específico (para análise e visualização, por exemplo), pode reduzir atritos e padronizar o fluxo de trabalho, em comparação ao uso isolado destes softwares.


\section{Motivação}


% Atualmente, existem diversos sistemas submarinos em operação nas Bacias de Campos e Espírito Santo que estão no final ou já ultrapassaram a metade de sua vida útil de projeto, o que torna ainda mais relevante uma análise de integridade avaliando possibilidade de extensão de vida operacional com critérios de cálculo validados.

No cenário mundial existe a tendência da indústria de óleo e gás de investimento em transformação digital em todas as áreas da cadeia, com desenvolvimento de práticas e ferramentas. Esse movimento levou ao surgimento de ferramentas específicas ao auxílio do profissional responsável pela análise, visualização, predição dos resultados de VIV em dutos em vão-livre~\cite{Mittal2017}. No entanto, pela especificidade dessas ferramentas, seu número ainda é reduzido, destacando-se apenas duas de nível comercial.


\subsection{Soluções comerciais existentes}


% \subsubsection{SAGE Profile}


Desenvolvido pela empresa Fugro, que atua no monitoramento de dutos submarinos, o SAGE Profile~\cite{sageprofile} é o software deles para análise de dutos submarinos. Por ser específica para este uso, esta aplicação representa avanços em relação à modelagem com um software de elementos finitos genéricos. No entanto, ela se limita a análise de elementos finitos, deixando a análise de fadiga a cargo do usuário. Além disso, a interação do usuário está limitada a interface gráfica (GUI\footnote{\textit{Graphical User Interface}: interface de usuário gráfica}, como na \autoref{fig:sageprofile}), o que dificulta a automação de tarefas corriqueiras.

\begin{figure}[!ht]
    \centering
    \caption{Interface gráfica do SAGE Profile.}\label{fig:sageprofile}
    \begin{subfigure}[t]{0.49\textwidth}
        \centering
        \includegraphics[width=\textwidth]{imagens/sage_profile_1}
    \end{subfigure}
    \hfill
    \begin{subfigure}[t]{0.49\textwidth}
        \centering
        \includegraphics[width=\textwidth]{imagens/sage_profile_2}
    \end{subfigure}
    \fonte{www.sage-profile.com.}
\end{figure}


% \subsubsection{Sesam for pipelines}


Em outra frente, tem-se a DNG-GL, que é uma referência mundial, entre outras áreas, em análise de dutos em vão-livre. Esta empresa é responsável pelo desenvolvimento da suíte Sesam for pipelines~\cite{dnvsesam} focados na análise de dutos submarinos.
Essa suíte consiste em 6 aplicações em VBA\footnote{Sigla para \textit{Visual Basic for Applications}, uma linguagem pela qual se pode customizar e estender aplicações \textit{desktop} da suíte Microsoft.}, dentre as quais a mais destacada é a \fatfree\footnote{Planilha Microsoft Office Excel desenvolvida pela \dnvf105 focada no cálculo de vida à fadiga de dutos submarinos.}, responsável pelo cálculo da vida à fadiga em si. Sendo desenvolvidas pela DNG-GL, as aplicações seguem as recomendações práticas propostas pela mesma --- que traz bastante confiabilidade nos resultados. Entretanto, apesar de conter uma aplicação para análise de comportamento mecânico, estas aplicações são simples, e estão muito aquém de uma solução completa para simulação assentamento do tudo no solo, como o SAGE Profile.

\begin{figure}[!ht]
    \centering
    \caption{Interface gráfica (planilha) da \fatfree.}\label{fig:fatfree}
    \includegraphics[width=0.8\textwidth]{imagens/fatfree}
    \fonte{Autor (2020)}
\end{figure}

Pelo exposto sobre o SAGE Profile e o Sesam for pipelines, vê-se que dentre as melhores ferramentas atuais para análise fadiga de duto em vão-livre ainda há espaço para melhorias, especialmente no que se refere a integração. Diante dessas limitações, é comum usar um software para a análise de elementos finitos genérico (como ABAQUS e ANSYS\footnote{https://www.ansys.com}) e realizam o cálculo de fadiga em folhas de cálculo pessoais ou comerciais, como a \fatfree.

Visando desenvolver uma nova ferramenta para suprir esta necessidade. Portanto, um \textit{framework} com o qual se possam construir aplicações para automatizar tarefas no fluxo de trabalho desde a análise de elementos finitos até análise de fadiga traria uma importante contribuição para o panorama atual.


\section{Objetivos}


Este trabalho tem como objetivo geral desenvolver um \textit{framework} para a análise de fadiga em dutos submarinos em vãos-livres, que permita um fluxo de trabalho que inclua um software de análise de elementos finitos e uma planilha de cálculo de vida à fadiga. A entrega de valor do \frame\ reside na automação de tarefas das etapas de pré e pós-processamento dos dados de entrada do ABAQUS e \fatfree, bem como a possibilidade de execução programática do destas ferramentas.
Além disso, este trabalho tem como objetivos específicos:

\begin{itemize}
    \item Contribuir para a metodologia de análise de fadiga em dutos por meio da criação de uma metodologia de seleção de modos de vibração;
    \item Modelar e implementar um \textit{framework} utilizando o paradigma da programação orientada a objetos, através da linguagem \textit{Python};
    \item Apresentar a aplicação de algumas das funcionalidades o \textit{framework} em um exemplo.
\end{itemize}


\section{Delimitação do trabalho}


Este trabalho descreve o processo de desenvolvimento de um \frame\ computacional para auxílio no processo de análise de vida à fadiga de dutos submarinos em vão-livre. O \frame\ desenvolvido não pretende implementar o processo de análise numérica de elementos finitos, nem o cálculo de vida à fadiga, estes ficaram a cargo de ferramentas comerciais de terceiros: ABAQUS e \fatfree, respectivamente.


\section{Organização do Trabalho}

O \autoref{chap:viv}: apresenta os conceitos e as formulação por trás da Vibração Induzida por Vórtice (VIV), e as recomendações dados pela referência técnica \dnvf105. O \autoref{chap:assentamento}: apresenta a modelagem computacional do comportamento de duto submarinos, os passos de carga, e os tipos de elementos empregados para modelar os elementos necessários a simulação. O \autoref{chap:software}: apresenta os aspectos da implementação computacional do \textit{framework} , bem como os princípios norteadores de algumas escolhas da, a escolha da linguagem e paradigma de programação, a estrutura proposta para os módulos e classe.

\thispagestyle{simple}
\chapter{Metodologia}\label{chap:metodologia}

Para alcançar os objetivos propostos neste trabalho, a metodologia de desenvolvimento foi baseada em 5 etapas, conforme ilustra a~\autoref{fig:fluxograma_metodologia}, a seguir:

\begin{figure}[!ht]
    \centering
    \caption{Fluxograma das etapas do trabalho.}\label{fig:fluxograma_metodologia}
    \includegraphics[width=\textwidth]{imagens/fluxograma_metodologia}
    \fonte{Autor (2021).}
\end{figure}

A primeira etapa consistiu em estudar o fluxo básico de trabalho de profissional, desde o acesso às informações até chegar nos resultados de fadiga. Essa etapa foi desenvolvida através de reuniões e oficinas ministrados por engenheiros da PETROBRAS que atuam na realização das análises de dutos em vão-livre (interessados diretos no desenvolvimento da ferramenta). Nessa etapa foram identificados as principais dificuldades do processo, destacando aqueles de maior potencial de automatização.

Em seguida, na segunda etapa, buscou-se conhecer os softwares que seriam integrados pela ferramenta: ABAQUS e FatFree. Com isso, pôde-se conhecer as oportunidades e limitações para integração de cada uma delas. %chktex 19

Na terceira etapa, fez-se a definição dos requisitos do sistema. Por exemplo: elaboração da especificação do fluxo dos dados na ferramenta, culminando na especificação de um arquivo de entrada para facilitar a estruturação dos dados para consumo pela aplicação. E assim chegou-se a um novo fluxo de trabalho que será apresentado mais adiante na~\autoref{sec:fluxo-com-ferramenta}.

Na quarta etapa, houve o desenvolvimento de uma série de \textit{scripts} em linguagem Python para automação de algumas tarefas.
Essa fase, de caráter exploratório, permitiu o desenvolvimento de pequenas ferramentas que podiam ter \textit{feedback} mais rápido, melhorando o entendimento dos requisitos e compressão da visão geral do fluxo de trabalho.
Com esse conjunto inicial de \textit{scripts}, fez-se então a modelagem dos módulos da aplicação com base nas diferentes funcionalidades previstas para a ferramenta.
Os códigos dos \textit{scripts} foram, então, reagrupados nesses módulos, que se comunicam por meio da utilização das diversas classes e respectivos métodos.

Finalmente, na quinta e última etapa, fez-se a validação da ferramenta usando dados de dutos reais da PETROBRAS\@.
No entanto, devido à confidencialidade destes dados, neste trabalho não serão apresentados os casos usados na validação, mas casos com dados fictícios para exemplificar as funcionalidades da ferramenta.
\chapter{Fundamentação teórica}

\chapter{Vibração induzida por vórtices em vãos livres}

Quando um fluido de baixa viscosidade encontra um obstáculo, forma-se uma camada limite.
Esta fina camada de fluido está sujeita aos efeitos das forças viscosas.
Nesta camada a velocidade do fluxo varia rapidamente, ficando cada vez mais lenta, formando um escoamento rotacional dentro da camada limite.
Para determinadas velocidades de escoamento, a camada limite se desprende do obstáculo, formando uma esteira de vórtices, conhecida como esteira de von Kármán~\cite{Currie2002}, conforme visto na Figura~\ref{fig:viv_shading}.
Como consequência direta do desprendimento de vórtices, surge uma força oscilatória transversal ao fluxo, que age sobre o obstáculo, resultando em oscilações verticais e horizontais~\cite{Nielsen2002}.

\begin{figure}[!ht]
    \centering
    \caption{Esteira de Von Kármán.}\label{fig:viv_shading}
    \includegraphics[width=0.6\textwidth]{imagens/viv_shading}
    \fonte{\citeonline{VandenAbeele2013}.}
\end{figure}

A frequência do desprendimento de vórtices causado por um fluxo normal ao obstáculo (o duto em vão livre, no caso em questão), é governado pelo número de Strouhal, diâmetro externo e velocidade de fluxo~\cite{Mork2003}.
O número de Strouhal pode ser obtido pela expressão $S_t = (f L) / V$, onde $f$ é a frequência de vórtices, $L$ é o comprimento característico e $V$ é a velocidade do fluxo.
Quando a velocidade do fluxo alcança uma das frequências naturais da estrutura, ela começa a vibrar e estas duas vibrações se correlacionam, causando vibrações de grande amplitude e grande dano (\textit{lock-in})~\cite{Mork2003}.

Como os dutos são geralmente modelados como cilindros, é importante entender como funciona o comportamento do fluxo de fluido ao redor dessa estrutura. Segundo \apudonline{Sumer1995}{Batchelor1967}, ao estudar vibrações de cilindros em corrente constante, inicia-se o desprendimento de vórtices quando o número de Reynolds, $R_e = (U\cdot D)/\nu$, é maior que $40$, onde $U$ é a velocidade do fluxo, $D$ é o diâmetro do cilindro e $\nu$ é a viscosidade cinemática~\cite{Sumer1995}.

O desprendimento de vórtices induz uma variação cíclica de forças no cilindro.
Assim, enquanto uma força de sustentação (\textit{lift force}) oscila à mesma frequência do desprendimento de vórtices, a força de arrasto (\textit{drag force}) oscila à duas vezes esta mesma frequência~\cite{Sumer1995}.
Estas forças oscilatórias, os vórtices, podem induzir vibrações na direção ortogonal ao fluxo, \textit{cross-flow} (CF), e na direção do fluxo, \textit{in-line} (IL), denominadas: vibrações induzidas por vórtices (VIV).

Os diversos dutos submarinos, que tem como objetivo o transporte de fluidos, seja entre o poço e a plataforma, entre plataformas etc., estão sujeitos ao fluxo intermitente de cargas ambientais.
Essas cargas, tornam-se um desafio ainda maior quando os dutos, instalados diretamente no irregular leito marinho, encontram-se em vãos livres~\cite{Fyrileiv1998}, como ilustrado na~\autoref{fig:freespan}.

\begin{figure}[!ht]
	\centering
    \caption{Duto em Vão livre e direções das oscilações.}\label{fig:freespan}
	\includegraphics[width=1\textwidth]{imagens/freespan}
    \\Fonte:~\citeonline{DNV2017}.
\end{figure}

Porém, vãos livres não aparecem apenas quando os dutos são instalados em leito irregular, mas também quando ocorre erosão posterior (\textit{scouring}\footnote{Erosão do solo marinho causada pela ação de ondas ou correntes. Caracteriza-se pela remoção de sedimentos com formação de cavidades ou canais.}), devido, por exemplo, a suportes artificiais.
Com o duto exposto à ondas e correntes, a parte não apoiada estará suscetível à VIV.\@
Caso a frequência de desprendimento alcance uma das frequências naturais do duto, esse poderá entrar em ressonância.
As excitações dinâmicas podem causar danos por fadiga, sendo importante identificar os corretos procedimentos de intervenção, seja no duto ou no leito marinho.

A \dnvf105 utiliza uma metodologia baseada em modelos de resposta a fim de avaliar a fadiga causada por VIV em dutos em vão livre.
Estes modelos representam relações empíricas entre a velocidade reduzida (\autoref{eq:viv-Vr}) e a amplitude de resposta adimensional, utilizadas para prever as amplitudes de vibração nas direções \textit{in-line} e \textit{cross-flow}~\cite{Mork2003,DNV2017}.
Além desta, a recomendação prática sugere também um método baseado no coeficiente de sustentação e nas curvas do coeficiente de massa adicional como função da amplitude de resposta adimensional e da frequência de vibração adimensional~\cite{DNV2017}.
Como terceira opção, a \dnvf105 indica o uso de fluidodinâmica computacional (CFD, na sigla em inglês) para escoamento turbulento ao redor dos dutos para avaliação do VIV.\@

A \dnvf105 considera dois modelos para estimar a resposta dinâmica em um vão livre: Modelo de Resposta (\textit{Response Model} - RM) e Modelo de Força (\textit{Force Model} - FM).
A escolha do modelo, segundo~\citeonline{Tura1994}, depende: (i) do comportamento dos carregamentos ambientais, isto é, quando há ressonância induzida por vórtice, aplica-se RM; e quando o comportamento do vão livre é afetado por carregamentos periódicos com pouca ou nenhuma amplificação dinâmica, aplica-se FM; (ii) da direção e tipo de fluxo, RM é aplicável na direção \textit{in-line} para corrente contínua e na direção \textit{cross-flow} para qualquer padrão de fluxo; o FM é aplicado na direção \textit{in-line} para carregamentos de onda direto.

% Caracteriza-se \textit{single mode} quando há um único valor para amplitude de corrente, sem que haja variações com a direção, e não houver presença de ondas.
A \dnvf105 pode ser aplicada para vãos únicos e múltiplos onde um modo de vibração é predominante (\textit{single-mode}), conforme a \autoref{fig:vaos_tipicos}.
Porém, a combinação de vãos de grande extensão e altas correntes, ou ainda vãos múltiplos, faz com que não apenas os modos fundamentais sejam ativados, mas também diversos outros modos de ordem mais alta (\textit{multi-mode}).

\begin{figure}[!ht]
	\centering
    \caption{Configurações típicas para vãos.}\label{fig:vaos_tipicos}
	\includegraphics[width=0.7\textwidth]{imagens/vaos_tipicos}
    \\Fonte:~\citeonline{Bai2014}.
\end{figure}

A primeira etapa do cálculo das frequências de vibração dos dutos em vão livre é o cálculo da força axial efetiva, $S_\mathit{eff}$, e os parâmetros de rigidez do solo $K_v$, $K_l$ calculados por meio das Equações~\ref{eq:viv-Kv} e~\ref{eq:viv-Kl}, sendo $K_\mathrm{v,s}$ um valor tabelado de acordo com o tipo de solo escolhido.

Com o intuito de demonstrar a formulação do modelo de resposta para o caso \textit{single mode} de um duto totalmente restringido, tem-se que a tensão axial efetiva é dada por
\begin{equation}
\label{eq:viv-Seff}
S_\mathit{eff} = H_\mathit{eff} - \Delta p_i A_i (1 - 2\nu) - A_s E \Delta T \alpha_E
\end{equation}
onde

\begin{tabular}{rl}
$H_\mathit{eff}$ & tensão efetiva de lançamento\\
$\Delta p_i$     & diferencial de pressão interna em relação ao lançamento\\
$A_i$            & área da seção transversal interna do duto de aço\\
$\nu$            & coeficiente de Poisson\\
$A_s$            & área da seção transversal externa do duto de aço\\
$E$              & módulo de elasticidade\\
$\Delta T$       & diferencial de temperatura em relação ao lançamento\\
$\alpha_E$       & coeficiente de expansão de temperatura
\end{tabular}

Em seguida, calcula-se a carga crítica de flambagem, definida como
\begin{equation}
\label{eq:viv-Pcr}
P_\mathit{cr} = (1 + \mathit{CSF}) C_2\pi^2 \mathit{EI}/L_\mathit{eff}^2
\end{equation}
com
\begin{equation}
\label{eq:viv-CSF}
\mathit{CSF} = k_c {\left(\frac{\mathit{EI}_\mathit{conc}}{\mathit{EI}}\right)}^{0,75}
\end{equation}
onde

\begin{tabular}{rl}
	$\mathit{CSF}$               & fator de rigidez do concreto\\
	$C_2$                        & coeficiente das condições de contorno\\
	$\mathit{EI}$                & rigidez à flexão do aço\\
	$L_\mathit{eff}$             & comprimento efetivo do vão\\
	$k_c$                        & constante empírica para a rigidez do concreto\\
	$\mathit{EI}_\mathit{conc}$  & rigidez à flexão do concreto
\end{tabular}

A constante empírica $k_c$ considera a deformação/deslizamento no revestimento anti-corrosão e as fraturas no revestimento de concreto.

O comprimento efetivo do vão, dado pelo produto do comprimento do vão por um fator de escala, é necessário visto que as condições de contorno nos ombros (\textit{shoulders}) que o duto se apoia estão entre \textit{pinned-pinned} e \textit{fixed-fixed}\footnote{Valores dos fatores $C_1$ a $C_6$ estão dispostos na Tabela 6-1~cite[p. 111]{DNV2017}, e só devem ser utilizados apenas para cenários \textit{single-span}.}.
Logo, temos que
\begin{equation}
\label{eq:viv-LeffL}
\frac{L_\mathrm{eff}}{L} =
\begin{cases}
	4,73 / (-0,066 \beta^2 + 1,02 \beta + 0,63)   & \mathrm{para} \beta \geq 2,7 \\
	4,73 / (0,036 \beta^2 + 0,61 \beta + 1)       & \mathrm{para} \beta <    2,7
\end{cases}
\end{equation}
com
\begin{equation}
\label{eq:viv-beta}
\beta = \log_{10}\left( \frac{K L^4}{(1 + \mathit{CSF})\mathit{EI}_\mathit{conc}} \right)
\end{equation}
onde $L$ é o comprimento real do vão e $K$ é a rigidez estática ou dinâmica do solo por unidade de comprimento.

Pode-se encontrar o módulo de Young do concreto a partir da expressão
\begin{equation}
\label{eq:viv-Econc}
E_\mathit{conc} = 10000 f_\mathit{cn}^{0,3}
\end{equation}
onde $f_\mathit{cn}$ é a resistência de fabricação do concreto.

Os parâmetros de rigidez do solo são calculado com base na DNVGL-RP-F114~\cite{DNVF114}.
A rigidez dinâmica do solo por metro na direção vertical (\textit{cross-flow}) é dada por
\begin{equation}
\label{eq:viv-Kv}
K_v = \frac{C_v}{1 - \nu_\mathit{soil}}\left(\frac{2}{3}\frac{\rho_s}{\rho}+\frac{1}{3}\right)\sqrt[]{D}
\end{equation}
e a rigidez dinâmica do solo por metro na direção lateral (\textit{in-line}) por
\begin{equation}
\label{eq:viv-Kl}
K_l = C_l (1+\nu_\mathit{soil})\left(\frac{2}{3}\frac{\rho_s}{\rho}+ \frac{1}{3}\right)\sqrt[]{D}
\end{equation}
onde

\begin{tabular}{rl}
	$C_v$               & fator de rigidez dinâmica do solo na direção vertical\\
	$C_l$               & fator de rigidez dinâmica do solo na direção longitudinal\\
	$\nu_\mathit{soil}$ & coeficiente de Poisson do solo\\
	$\rho_s$            & massa específica do duto\\
	$\rho$              & massa específica da água deslocada\\
	$D$                 & diâmetro externo do duto (incluindo revestimento)
\end{tabular}

Caso não seja um dado advindo das medições, ou estimado analiticamente, é necessário calcular a deflexão estática \textit{mid-span}, que é dada por
\begin{equation}
\label{eq:viv-deflex}
\delta = C_6 \frac{q L_\mathit{eff}^4}{\mathit{EI} (1 + \mathit{CSF})} \frac{1}{S_\mathit{eff}/P_\mathit{cr}}
\end{equation}
onde $C_6$ é um coeficiente da condição de contorno e $q$ é o peso submerso.

A frequência natural fundamental, a ser definida para as direções \textit{in-line} e \textit{cross-flow}, pode ser aproximada a partir de
\begin{equation}
\label{eq:viv-f1}
f_1 \approx C_1 \sqrt{1 + \mathit{CSF}}\sqrt{\frac{\mathit{EI}}{m_e} L_\mathit{eff}^4} \left(1 + \frac{S_\mathit{eff}}{P_\mathit{cr}} + C_3 {\left(\frac{\delta}{D}\right)}^2\right)
\end{equation}
onde $C_1$ e $C_3$ são coeficientes de condições de contorno e $m_e$ é a massa efetiva, incluindo a massa estrutural, massa do fluido interno e massa adicional.

Desta forma, o efeito da massa adicional pode ser modelado a partir do coeficiente de massa adicional ($C_a$), que pode ser aplicado para superfícies suaves ou rugosas do duto e deve ser aplicada para frequência natural da água parada, sendo calculado da seguinte forma
\begin{equation}
\label{eq:viv-Ca}
C_a =
\begin{cases}
	0,68 + \frac{1,6}{1 + 5 (e/D)} & \mathrm{para} e/D < 0,8 \\
	1                              & \mathrm{para} e/D \geq 0,8
\end{cases}
\end{equation}
onde $e$ corresponde ao \textit{gap} do vão, isto é, a distância entre o duto e o solo marinho.

Além disto, podem-se calcular também a amplitude máxima de tensão para o diâmetro unitário para os modos fundamentais \textit{in-line} ($\mathit{IL}$) e \textit{cross-flow} ($\mathit{CF}$) assim
\begin{equation}
\label{eq:viv-Ailcf}
A_{\mathit{IL}/\mathit{CF}, 1}^{\max} = 2 C_4(1 + \mathit{CSF})\frac{D E r}{L_\mathit{eff}^2}
\end{equation}
em que $r$ é uma coordenada radial da seção transversal do duto e $C_4$ é um coeficiente de condição de contorno.


Por fim, finaliza-se esta etapa com o cálculo do fator de redução para corrente, $R_C$, que será aplicado na velocidade de referência, sendo calculado assim
\begin{equation}
\label{eq:viv-R_C}
R_C(z) = R_c \frac{\ln(z)-\ln(z_0)}{\ln(z_r)-\ln(z_0)}
\end{equation}
com o fator de referência dado por
\begin{equation}
\nonumber
R_c = \sin(\theta_\mathit{rel})
\end{equation}
onde $z$ é a altura acima do solo, $z_0$ é o parâmetro de rugosidade, $z_r$ é a altura de medição de referência e $\theta_\mathit{rel}$ é o ângulo formado entre a corrente e o duto.

Podemos utilizar os resultados das equações acima para a construção dos modelos de resposta relacionando a velocidade do fluxo com a amplitude de vibração.
Pela \dnvf105, as vibrações \textit{in-line} e \textit{cross-flow} devem ser consideradas em modelos de resposta separados.


\section{Modelo de resposta \textit{in-line}}\label{sec:modelo-resposta-inline}

O parâmetro de estabilidade, $K_S$, representa o amortecimento para uma dada forma modal, sendo obtido a partir da equação
\begin{equation}
\label{eq:viv-Ks}
K_S = \frac{4 \pi m_e \zeta_T}{\rho_w D^2}
\end{equation}
em que $\rho_w$ é a densidade da água e $\zeta_T$ é a taxa de amortecimento modal total.
Aplica-se um fator de segurança ao parâmetro de estabilidade, $K_\mathit{Sd} = K_S/\gamma_k$, sendo $\gamma_k$ o fator de segurança no parâmetro de estabilidade.

Em seguida, deve-se calcular os fatores de correção para considerar a turbulência e o ângulo de ataque do fluxo
\begin{equation}
\label{eq:viv-Riot}
\begin{aligned}
    R_{I\theta,1} &= 1 - \pi^2\left(\frac{\pi}{2} - \sqrt{2 \theta_\mathit{rel}}\right)(I_c - 0,03) & \quad \mbox 0 \leq R_{I\theta,1} \leq 1\\
    R_{I\theta,2} &= 1 - \frac{I_c - 0,03}{0,17}                                                    & \quad \mbox  0 \leq R_{I\theta,2} \leq 1
\end{aligned}
\end{equation}
onde $I_c$ é a intensidade de turbulência.

Segundo a \dnvf105 o fluxo pode ser dividido em duas zonas: (i) uma zona exterior, distante do solo marinho, onde velocidade de corrente média e a turbulência variam muito pouco na direção horizontal, e (ii) uma zona interior, onde a velocidade de corrente média e a turbulência têm variações consideráveis na direção horizontal. Uma vez que as medições da corrente são realizadas na zona exterior, fora da camada limite, a velocidade de corrente no duto pode ser aproximada a partir da equação
\begin{equation}
\label{eq:vel-corrente}
U_c = R_c U(z_r) \frac{\ln{(e+D/2)} - \ln(z_0)}{\ln (z_r)- \ln (z_0)}
\end{equation}
em que $U(z_r)$ é a velocidade da corrente na altura de referência.

Uma vez encontrada a velocidade da corrente na zona interior, isto é, próxima do solo, a velocidade reduzida pode ser calculada assim
\begin{equation}
\label{eq:viv-Vr}
V_R = \frac{U_c + U_w}{f_n D}
\end{equation}
onde $U_w$ é a velocidade de fluxo induzida por onda e $f_n$ é a frequência natural de amplitude.

A amplitude de resposta \textit{in-line} depende da  (\autoref{eq:viv-Vr}), $V_R$, do parâmetro de estabilidade, $K_\mathit{Sd}$, da intensidade da turbulência, $\mathit{I}_c$, e do ângulo do fluxo, $\theta_\mathit{rel}$.
O modelo de resposta pode então ser construído através do conjunto de equações a seguir

\begin{equation}
\label{eq:viv-Vronset}
\frac{A_{Y,1}}{D} = \max\left(0,18 \left(1 - \frac{K_\mathit{Sd}}{1,2}\right) R_{I\theta,1} ~,~\frac{A_{Y,2}}{D}\right)\\
\end{equation}

\begin{equation}
\frac{A_{Y,2}}{D} = 0,13 \left(1 - \frac{K_\mathit{Sd}}{1,8}\right) R_{I\theta,2}\\
\end{equation}

\begin{equation}
V_{R,\mathit{onset}}^\mathit{IL} =
\begin{cases}
\frac{1}{ \gamma_{\mathit{on}, \mathit{IL}} }                 & \mathrm{para} K_\mathit{Sd} < 0,4\\
\\
\frac{0,6 + K_\mathit{Sd}}{\gamma_{\mathit{on}, \mathit{IL}}} & \mathrm{para} 0,4 \leq K_\mathit{Sd} < 1,6 \\
\\
\frac{2,2}{\gamma_{\mathit{on}, \mathit{IL}}}                 & \mathrm{para} K_\mathit{Sd} \geq 1,6
\end{cases}
\end{equation}

\begin{equation}
V_{R, \mathit{end}}^\mathit{IL} =
\begin{cases}
4,5 - 0,8 K_\mathit{Sd} & \mathrm{para} K_\mathit{Sd} < 1,0 \\
3,7                     & \mathrm{para} K_\mathit{Sd} \geq 1,0
\end{cases}
\end{equation}

\begin{equation}
V_{R, 1}^\mathit{IL} = 10 \left(\frac{A_{Y, 1}}{D}\right)+ V_{R,\mathit{onset}}^\mathit{IL}\\
\end{equation}

\begin{equation}
V_{R, 2}^\mathit{IL} =  V_{R, \mathit{end}}^\mathit{IL} - 2 \left(\frac{A_{Y, 2}}{D}\right)\\
\end{equation}

onde $\gamma_{\mathit{on}, \mathit{IL}}$ é o fator de segurança para velocidade de corrente inicial \textit{in-line} e $\frac{A_Y}{D}$ é a amplitude \textit{in-line} normalizada. Com esses valores calculados, pode-se construir a curva que relaciona velocidade reduzida e amplitude de vibração para diâmetro unitário, semelhante a \autoref{fig:viv-responsemodelil}.

\begin{figure}[!ht]
    \centering
    \caption{Curva de modelo de resposta \textit{in-line}.}\label{fig:viv-responsemodelil}
    \includegraphics[width=0.8\linewidth]{imagens/response_model_IL}
    \\Fonte:~\citeonline{DNV2017}.
\end{figure}

Conforme observado na \dnvf105, a resposta de amplitude de um duto vibrando na direção \textit{in-line}, contempla regiões com velocidade de corrente entre 1,0 e 4,5.
Temos então que a resposta na direção longitudinal depende dos parâmetros de velocidade de corrente, estabilidade, intensidade de turbulência e do ângulo entre a corrente e o duto.
Percebe-se que, à medida em que o parâmetro de estabilidade aumenta, a amplitude de resposta tende à diminuir, uma vez que este é proporcional ao amortecimento do sistema~(\autoref{eq:viv-Ks}).


\section{Modelo de resposta \textit{cross-flow}}

Para o modelo de resposta \textit{cross-flow} também  é necessário calcular um conjunto de parâmetros. Dessa vez, inicia-se com o cálculo do fator de correção para considerar a proximidade do duto com o solo
\begin{equation}
\label{eq:viv-Psi}
\Psi_{\mathit{proxi}, \mathit{onset}} =
\begin{cases}
\frac{1}{5}\left(4 + 1,25\frac{e}{D} \right) & \mathrm{para}~\frac{e}{D} < 0,8\\
1                                            & \mathrm{caso~contr\acute{a}rio}
\end{cases}
\end{equation}

Caso o duto esteja localizado próximo ou em trincheiras é necessário considerar o fator de correção específico

\begin{equation}
\label{eq:viv-Psitren}
\Psi_{\mathit{trench}, \mathit{onset}} = 1 + 0,5\frac{\Delta}{D}
\end{equation}
onde $\Delta$ é a profundidade da trincheira.

O número de Keulegan-Carpenter é definido como
\begin{equation}
\label{eq:viv-KC}
\mathit{KC} = \frac{U_w}{f_w D}
\end{equation}
onde $f_w = \frac{1}{T_u}$ é o período de cruzamento da frequência de onda.

A razão de velocidade de fluxo de corrente é dada por
\begin{equation}
\label{eq:viv-alfa}
\alpha = \frac{U_c}{U_c + U_w}
\end{equation}

A partir dessas equações, se pode construir o modelo de resposta \textit{cross-flow} através do conjunto de equações a abaixo
\begin{equation}
\label{eq:viv-azdj}
\frac{A_{Z,1}}{D} =
\left\{
\begin{array}{ccc}
0,9                                                      & \alpha > 0,8   &         \frac{f_{n+1,CF}}{f_{n,CF}} <   1,5 \\
0,9 + 0,5 \left(\frac{f_{n+1,CF}}{f_{n,CF}} - 1,5\right) & \alpha > 0,8   & 1,5 \le \frac{f_{n+1,CF}}{f_{n,CF}} \le 2,3 \\
1,3                                                      & \alpha > 0,8   &         \frac{f_{n+1,CF}}{f_{n,CF}} >   2,3 \\
0,9                                                      & \alpha \le 0,8 &        \mathit{KC} >   30 \\
0,7 + 0,01 (\mathit{KC} -10)                             & \alpha \le 0,8 & 10 \le \mathit{KC} \le 30 \\
0,7                                                      & \alpha \le 0,8 &        \mathit{KC} <   10
\end{array}
\right.
\end{equation}

\begin{equation}
\frac{A_{Z,2}}{D} = \frac{A_{Z,1}}{D}
\end{equation}

\begin{equation}
V_{R,\mathit{onset}}^\mathit{CF} = \frac{3 \cdot \Psi_{\mathit{proxi}, \mathit{onset}} \cdot  \Psi_{\mathit{trench}, \mathit{onset}}}{\gamma_{\mathit{on}, \mathit{CF}}}
\end{equation}

\begin{equation}
V_{R,\mathit{end}}^\mathit{CF} = 16
\end{equation}

\begin{equation}
V_{R, 1}^\mathit{CF} = 7 - \frac{7 - V^\mathit{CF}_{R, \mathit{onset}}}{1,15} \left(1,3 - \frac{A_{Z,1}}{D}\right)
\end{equation}

\begin{equation}
V_{R, 2}^\mathit{CF} = V^\mathit{CF}_{R, \mathit{end}} - \frac{7}{1,3} \frac{A_{Z, 1}}{D}\\
\end{equation}

Com esses valores calculados, pode-se construir a curva que relaciona velocidade reduzida e amplitude de vibração para diâmetro unitário, semelhante a \autoref{fig:viv-responsemodelcf}.

\begin{figure}[!ht]
    \centering
    \caption{Curva de modelo de resposta \textit{cross-flow}.}\label{fig:viv-responsemodelcf}
    \includegraphics[width=0.8\linewidth]{imagens/response_model_CF}
    \\Fonte:~\citeonline{DNV2017}.
\end{figure}


\section{\label{sec:multimode}Resposta \textit{multi-mode}}

A resposta do vão livre pode ser dada em função de uma coordenada $x$ ao longo da direção longitudinal do duto.
Para cada combinação relevante de estado de mar e velocidade de corrente, um número de modos pode ser excitado simultaneamente na mesma direção, dando origem a uma resposta \textit{multi-mode}.
Todavia, o número de modos que responderão e o quanto cada modo contribuirá para o dano por fadiga dependerá da velocidade do fluxo, da posição no eixo $x$ e da competição com outros modos.

A \dnvf105 define três diferentes tipos de modos:
\begin{description}
	\item[Modos ativos] são os modos que podem ser excitados por VIV. Com base no itém 2.3.3 da DNVGL-RP-F105, os critério para definição de para que um modo \textit{in-line}, com frequência $f_{IL,j}$, ou \textit{cross-flow}, com frequência $f_{CF,j}$, seja considerado ativo é:

    \begin{equation}
        \begin{aligned}
        f_{I L, j} & \leq \frac{U_{\text{extreme}} \gamma_{f,IL}}{V_{R,\text{onset}}^{IL} D} \\
        f_{C F, j} & \leq \frac{U_{\text{extreme}} \gamma_{f,CF}}{2D}
        \end{aligned}
    \end{equation}
    sendo $\gamma_{f,IL}$ e $\gamma_{f,CF}$ coeficientes de segurança, variando de 1 a 1,3 a depender da classe de segurança e nível de
    definição do vão livre (item 2.7.2 da DNVGL-RP-F105).
    Um modo que não passível de ativação pode ser totalmente desconsiderado nas análises em todos os pontos e velocidades de fluxo.

    \item[Modos participantes] são modos ativos que têm amplitude de tensão relevante em um, ou ambos os lados, de um ponto x. Para que um modo j seja considerado participante no vão, é necessário que a seguinte condição (presente no item 4.3.3) seja atendida:
    $$
    \left|A_{\textbf{IL/CF}, j}(x)\right| \geq \frac{A_{IL/CF}^{\max}}{10} \text{ para algum } x \in (x_{\text{start},j}, x_{\text{end}, j})
    $$
    sendo
    $$
    A_{IL/CF, j}(x) = (1+CSF) D \kappa_{j}(x) E r
    $$
    onde

    \begin{tabular}{rl}
        $\mathit{CSF}$                            & fator de rigidez do concreto\\
        $\kappa_{j}(x)$                           & curvatura do modo na posição $x$\\
        $E$                                       & módulo de elasticidade\\
        $r$                                       & coordenada radial da seção transversal do duto\\
        $(x_{\text{start},j}, x_{\text{end}, j})$ & intervalo de influência do modo
    \end{tabular}

	\item [Modos contribuintes] são modos participantes que deve satisfizer um dos seguintes critérios:

        \begin{itemize}
    	\item direção \textit{cross-flow}: ${(A_Z/D)}_j \geq 0,1{(A_Z/D)}_{\max}$

        \item direção \textit{in-line}: $S_{\mathit{IL}, \mathit{j}}^{P}(x) \geq 0,1 S_\mathit{IL}^{\max}(x)$
        \end{itemize}

	onde ${(A_Z/D)}_j$ é a amplitude VIV normalizada para o j-ésimo modo, ${(A_Z/D)}_{\max}$ é a amplitude VIV normalizada para o modo \textit{cross-flow}  dominante, $S_{\mathit{IL}, \mathit{j}}^{P}(x)$ é a amplitude de tensões de reposta preliminar para o j-ésimo modo \textit{in-line} e $S_\mathit{IL}^{\max}(x)$ é a amplitude de tensões de resposta associadas ao modo \textit{in-line} dominante.

\end{description}

Baseado nos modelos de resposta \textit{single-mode}, podemos calcular as amplitudes do VIV para todos os modos (\autoref{procedure_il}).
Assim, precisamos calcular VIV \textit{cross-flow} e \textit{in-line} para cada velocidade de corrente, estado de mar e em cada ponto com se os seguintes procedimentos:

\begin{itemize}

\item VIV \textit{cross-flow}\label{procedure_il}

\begin{enumerate}
\item Identifica-se todos os modos participantes (\textit{single} ou \textit{multi location})

\item Com o modelo de resposta \textit{cross-flow}:
	\begin{enumerate}
    \item Calcula-se a amplitude VIV normalizada para cada modo ${(A_Z/D)}_j$

	\item Identifica-se o modo dominante, isto é, ${(A_Z/D)}_{\max}$

    \item Identificam-se os modos fracos $0,1{(A_Z/D)}_{\max} \leq {(A_Z/D)}_j \leq {(A_Z/D)}_{\max}$

    \item Desconsidera-se os modos irrelevantes: ${(A_Z/D)}_j$ < $0,1{(A_Z/D)}_{\max}$
    \end{enumerate}

\item Usando o modelo de resposta para baixos valores de Keulegan-Carpenter (\textit{low Keulegan Carpenter flow regime} - LKCR), calcula-se ${(A_Z/D)}_j$ para cada modo.

\item Determina-se a resposta de tensão combinada:
    $$
    S_{\mathit{comb}, \mathit{CF}} = \max\left( S_{\mathit{comb}, \mathit{CF}}^\mathit{RM} ~,~ S_{\mathit{comb}, \mathit{CF}}^\mathit{LKCR} \right)
    $$

\item Determina-se a frequência de contagem de ciclos:

    $$
    f_{\mathit{cyc}, \mathit{CF}} =
    \begin{cases}
    f_{\mathit{cyc}, \mathit{CF}}^\mathit{LKCR}, & S_{\mathit{comb},\mathit{CF}}^\mathit{RM}(x)    < S_{\mathit{comb},\mathit{CF}}^\mathit{LKCR}(x) \\
    f_{\mathit{cyc}, \mathit{CF}}^\mathit{RM},   & S_{\mathit{comb},\mathit{CF}}^\mathit{RM}(x) \geq S_{\mathit{comb},\mathit{CF}}^\mathit{LKCR}(x)
    \end{cases}
    $$

\end{enumerate}

\item VIV \textit{in-line}

\begin{enumerate}
    \item Identifica-se todos os modos participantes (\textit{single} ou \textit{multi location})

    \item Com o modelo de resposta \textit{in-line}:

    \begin{enumerate}
    	\item Calcula-se a amplitude VIV normalizada para cada modo ${(A_Y/D)}_j$

    	\item Identifica-se o modo dominante, isto é, o modo com $S_\mathit{IL}^{\max}(x)$

    	\item Identificam-se potenciais modos fracos: $0,1 S_\mathit{IL}^{\max}(x) \leq S_{\mathit{IL}, \mathit{j}}^{P}(x) \leq S_\mathit{IL}^{\max}(x)$

    	\item Desconsideram-se os modos irrelevantes: $S_{\mathit{IL}, \mathit{j}}^{P}(x) < 0,1 S_\mathit{IL}^{\max}(x)$
    \end{enumerate}

        \item Reduzir os modos fracos.
        Para VIV \textit{in-line}, dois modos adjacentes podem competir se suas frequências forem próximas, ou agir de forma independente se estiverem distantes.
        A \dnvf105 define que os modos competem se a razão entre as frequências é menor que 2, isto é, $\frac{f_\mathit{n+1}}{f_n} < 2$.
        Em modos adjacentes considera-se que apenas o ``vencedor'' da competição pode ter máxima amplificação, enquanto a amplificação do modo ``perdedor'' é reduzida à metade. É interessante ressaltar que modos que não competem não têm redução.

        \item Calcular o intervalo de tensões \textit{in-line} excitados pelo modo \textit{cross-flow} dominante $S_{\mathit{\textit{cross-flow}}-\mathit{IL}}(x)$.

        Para cada ponto e cada modo, calcula-se o intervalo de tensões induzido por VIV \textit{in-line} para os modos contribuintes:
            \[S_{\mathit{IL}, \mathit{j}}^\mathit{RM}(x) = S_{\mathit{IL}, \mathit{j}}^{P} \cdot 0,5^{\beta_j (x)}\]

        Assume-se que apenas o modo \textit{cross-flow} dominante é capaz de contribuir para o movimento \textit{in-line} induzido pelo modo transversal.
        Desta forma, o modo \textit{in-line} participante cuja frequência natural é próxima a duas vezes a resposta \textit{cross-flow} dominante é escolhido como candidato a VIV \textit{in-line} induzido por \textit{cross-flow}.

        % \[\mid f_{\mathit{IL}, \mathit{k}}^\mathit{part} - 2 \cdot f_{\mathit{\textit{cross-flow}-RES}, \mathit{i}} \mid\]

        A amplitude de tensões \textit{in-line} excitados pelo modo \textit{cross-flow} dominante é dado por:

       	    \[S_{\mathit{CF-IL}}(x) = 0,8 \cdot A_{\mathit{IL}, \mathit{k}}~(x) \cdot {\left(\frac{A_{z}}{D}\right)}_{\max}~\cdot~R_k \cdot \gamma_s\]

    \item  Escolher o maior $S_\mathit{IL}^\mathit{RM}(x)$ e $S_{\mathit{CF-IL}}(x)$ para cada modo;

    \item Determinar a faixa de resposta de tensão combinada,

    \[S_{comb,IL}(x)=\sqrt{\sum_{j=1}^{m_{aig}}{\left(S_{IL,j}(x)\right)}^{2}}\]

    e a frequência de contagem de ciclos,

    \[f_{cyc,IL}(x)=\sqrt{\sum_{j=1}^{m_{\text{ug}}}{\left(f_{\text {IL}, j}^{con} \cdot \frac{S_{IL,j}(x)}{S_{comb,IL}(x)}\right)}^{2}}\]
    \end{enumerate}
\end{itemize}

\chapter{Modelagem de assentamento de dutos submarinos}
\label{chap:assentamento}

O assentamento de dutos submarinos tem sido o núcleo da engenharia \textit{offshore} por meio século. Vários métodos e técnicas tem sido desenvolvidas e usadas para dutos submarinos \cite[]{Ivi2016}.

Após que o traçado de uma rota ótima, simular o processo do assentamento de duto é uma das tarefas mais desafiadoras. Implementar a instalação de pipeline em um pacote de elementos finitos de uso geral pode ser um trabalho demorado e tedioso, em especial ao importar grandes quantidades de dados do leito marinho. Na maioria das vezes, são necessárias técnicas avançadas de script para definir o perfil do leito do mar, selecionar a rota ideal do pipeline e simular o processo de assentamento. Além disso, os modelos constitutivos disponíveis para interação tubo-solo podem não estar de acordo com os padrões da indústria.
\chapter{Desenvolvimento do \frame}\label{chap:software}

O levantamento dos requisitos de um sistema é o elemento que fornece elementos que deve nortear uma série de decisões a serem tomadas no seu desenvolvimento. Primeiramente, apresenta-se aqui o fluxo de trabalho tradicional para análise de fadiga.

\section{Fluxo de avaliação de vida à fadiga em dutos em vão livre}\label{chap:workflow}


Baseado nos estudos e \textit{oficinas} realizados para o desenvolvimento deste trabalho. Pôde-se estabelecer que a análise de vida a fadiga em dutos em vão livre compreende o fluxograma apresentado na \autoref{fig:fluxograma}.

\begin{figure}[!ht]
    \centering
    \caption{Fluxo de avaliação de vida à fadiga em dutos em vão livre.}\label{fig:fluxograma}
    \includegraphics[width=\textwidth]{imagens/fluxograma.pdf}
    \fonte{Autor (2020)}
\end{figure}

A seguir, uma breve descrição de cada item:

\begin{enumerate}[label=(\arabic*)]
    \item Nesta etapa, o profissional reúne as informações básicas para construção dos modelos e outros dados usados em cálculos posteriores. Citadamente, temos aqui: os as cotas do perfil do duto e batimetria obtidas na inspeção, geometria e composição das camadas que compõem sessão do duto, parâmetros do solo, constantes físicas e coeficientes de segurança, posição e tipos de suportes ao longo do duto. Essa tarefa envole olhar uma série de documentos (\texttt{.doc}, \texttt{.pdf}, etc) em busca desses valores, dispostos de forma não estruturada. Quando estruturados, em forma de arquivos CSV ou planilhas, por exemplo, é necessário ainda manipular esses dados a fim de extrair somente a informação necessária ou convertê-las no formato apropriado. Um exemplo disso são os dados de batimetria, que precisam convertidos nas coordenadas dos nós de uma malha de elementos finitos.
    \item 
\end{enumerate}{}


\section{Fluxo de avaliação de vida à fadiga com uso do \frame}


O \frame\ proposto tem como requisito atender o fluxo apresentado na \autoref{sec:workflow}, automatizando certas etapas da análise de vida a fadiga. Os itens coloridos são cobertos pela implementação \frame, e os itens  branco são realizados em aplicações externas.


\begin{figure}[!ht]
    \centering
    \caption{Fluxo de operação proposto para o \frame.}\label{fig:workflow}
    \includegraphics[width=\textwidth]{imagens/fluxograma_automatizado}
\end{figure}

De forma mais detalhada, a ferramente deve:

\begin{enumerate}[label= (\arabic*)]
    \item A partir de um arquivo de entrada com informações do modelo, criar arquivos de entrada para o ABAQUS (.inp) que reproduza todo o processo de simulação do comportamento do duto apresentado (\autoref{sec:assentamento}).
    \item Submeter o arquivo gerado para análise no ABAQUS. No caso de haver a colocação de suportes, a simulação é executada em duas partes: a primeira com os passos de carga anteriores a passo da colocação dos suportes, e a segunda com o passo da colação dos mesmos e os passos de carga seguintes.
    \item Processar os arquivos de saída do ABAQUS (.odb) extraindo as informações relevantes como a configuração deformada, modos de vibração, etc., gerando arquivos em outros formatos de fácil leitura para pós-processamento, tanto por este \frame, quanto por outros softwares.
    \item Pós-processar as informações gerando gráficos e relatórios relevantes para as tomadas de decisão do usuário quanto ao projeto. Esse é o requisito mais crítico, uma vez que é fundamental o entendimento sobre a análise de duto em vão-livre. Entre as tarefas que fazem parte deste item está a automação da escolha dos modos de vibração ativos e relevantes e para cada vão de interesse --- a qual deve ser norteada pelos aspectos discutidos na \autoref{sec:multimode} --- e a manipulação da FatFree.
    \item Ativar o processo de cálculo de fadiga no arquivo preenchido no passo anterior.
    \item Capturar os resultados no arquivo \texttt{.xls} da \fatfree, que agora contém os resultados do cálculo de vida a fadiga, e apresentá-los na forma de gráficos e relatórios.
\end{enumerate}

\section{Linguagem de programação}

Python\footnote{https://www.python.org} foi a linguagem de programação adotada. Além de ser uma linguagem interpretada de alto nível Orientada a Objeto --- que permite um alto índice de reaproveitamento de código --- e da sintaxe simples. \citeonline{Rao2018} apresenta algumas das principais vantagens que destaca a linguagem para este tipo de aplicação:

\begin{itemize}
    \item Disponibilidade de bibliotecas para aplicações cientificas contemplando manipulação de matrizes (Numpy), funções matemáticas (SciPy), manipulação de dados em forma tabular (Pandas), criação de gráficos interativos (Matplotlib e Bokeh).

    \item Suporte para automação de tarefas. Os recursos de \textit{script} internos do Python e vários pacotes têm um forte suporte à automação de tarefas. A automação de tarefas repetitivas e a realização do registro de dados são fáceis e requerem pouco esforço. O ABAQUS, por exemplo, permite modelagem e acesso a informações em arquivos de saída via Python. A biblioteca xlwings permite manipulação de planilhas Excel, a exemplo da FatFree.

    \item Pacotes Python como Django e Flask tornam possível desenvolver e usar o Python como uma API\footnote{Na programação de computadores, uma Interface de Programação de Aplicativos (\textit{Application Programming Interface}---API) é um conjunto de definições de sub-rotinas e ferramentas para a criação de software. Em termos gerais, é um conjunto de métodos de comunicação claramente definidos entre vários componentes.} com um \textit{front-end} da web. Essa funcionalidade é particularmente útil para reaproveitamento do \frame\  em outras aplicações.
\end{itemize}


\section{Pacotes e classes}


Para implementação do fluxo de trabalho proposto para o \frame, fez-se a implementação de módulos para lidar com cada contexto específico. Em python, módulos podem ser quaisquer arquivos com extensão \texttt{.py}. Estes módulos podem ser agrupados em pacotes, que são pastas que, além dos módulos, contém um arquivo \texttt{\_\_init\_\_.py}. No \frame\ têm-se os seguintes pacotes: % chktex 21

\begin{itemize}
    \item \texttt{analysis}: pacote principal responsável orquestrar o fluxo de trabalho do \frame\ desde o processamento dos dados de entrada, geração dos arquivos para o ABAQUS e os pós-processamentos.

    \item \texttt{odb\_handler}: responsável por lidar com os arquivos de saída do ABAQUS (odb) e guardar os dados relevantes em arquivos com formatos de fácil manipulação (CSV, JSON, etc\ldots).

    \item \texttt{mode\_selector}: pacote responsável pela estratégia de seleção automática de modos de vibração para cada vão e manipulação dos dados associados vãos e seus respectivos modos.

    \item \texttt{dnv}: pacote que implementa os cálculos dos modelos de resposta da \dnvf105 e manipula a planilha FatFree por meio da biblioteca xlwings.

    \item \texttt{plots}: pacote responsável pro agregar as funções de geração de gráficos dos resultados.
\end{itemize}

Já dentre as principais classes estão:

\begin{itemize}
    \item \texttt{Model}: classe que contém as informações do modelo do problema.
    A classe armazena todas as informações para construção dos arquivos \texttt{.inp}, isto é, dados de batimetria, material, geometria do duto, coeficientes de segurança, entre outros.
    A instanciação dessa classe deve ocorrer mediante o processamento de um arquivo principal de entrada com esses dados, em formato JSON (\autoref{apendice:json}).

    \item \texttt{Inp}: lida com a escrita modularizada de arquivos de entrada  para o ABAQUS. A proposta é que se crie um arquivo principal que terá inclusão de outros arquivos acessórios que terão as informações específicas de cada aspecto da modelagem: batimetria, passos de carga, etc.

    \item \texttt{ModeShape}: classe que representa um modo de vibração (\textit{mode shape}).

    \item \texttt{Span}: classe que representa um vão do duto. Uma vez que análise de fadiga é feita por vão, é nesta que são implementados os métodos responsáveis pela seleção dos modos de vibração, que são ligados à classe por um dos seus atributos.

    \item \texttt{Plot}: facilita a criação dos gráficos devido a implementação de métodos que permitem desde alterar os componentes mais comuns das figuras, como títulos dos eixos e legendas, até combinar gráficos e formar \textit{dashboards} com gráficos integrados.
\end{itemize}

Uma forma gráfica de apresentar essas entidades e seus relacionamentos é por meio do diagrama de diagrama UML\footnote{Sigla para \textit{Unified Modeling Language}: Linguagem Unificada de Modelagem é uma linguagem padrão para modelagem orientada a objetos}\cite{infoescolauml}. Para o caso do \frame\ aqui desenvolvido, a \autoref{fig:UML} apresenta esses módulos e classes, suas relações de pertencimento e dependência, e os principais métodos e atributos das classes.

\begin{figure}[!ht]
    \centering
    \caption{Diagrama UML dos módulos e classes.}\label{fig:UML}
    \includegraphics[width=\textwidth]{imagens/UML}
    \fonte{Autor (2019)}
\end{figure}


\chapter{Aplicações e resultados\label{chap:aplicacoes}}

Este capítulo apresenta um exemplo básico de aplicação da ferramenta, apresentando as funções principais para geração de modelos (.inp) e para simulação no ABAQUS, chamada do ABAQUS, pós-processamento e visualização dos resultados. Serão apresentados tanto os resultados obtidos com a simulação quanto os códigos básicos utilizados para geração desses resultados, bem como trechos de código demonstrando a forma de utilização das principais funcionalidades implementadas.

Embora não haja restrições da ferramenta quanto ao perfil da batimetria, este exemplo será um caso de uma batimetria simples, uma vez que o intuito é apresentar modelo cujos resultados (mais especificamente, os valores das frequências obtidas pela análise modal no ABAQUS) podem ser validados com os resultados previstos (calculados pela \fatfree).
Vale salientar que os dados utilizados são de domínio público ou fictícios, mas não representam um caso real, até porque que esse tipo de dado é mantido sob sigilo devido a aspectos relacionados à segurança, competitividade tecnológica e propriedade intelectual das empresas do setor.


\section{Exemplo: análise de vida à fadiga de um pequeno trecho de duto\label{sec:model-exemplo}}


O modelo consiste um pequeno trecho de duto com comprimento total de 1000~m com um vão de 17~m posicionado no centro. A grande extensão do duto a direita e à esquerda do vão é usada simplesmente para reduzir a influência das condições de contorno nas extremidades.
% As principais características geométricas e físicas do duto estão apresentadas na
% TODO \autoref{tab:propriedades}.

% \begin{table}[ht]
% 	\renewcommand{\arraystretch}{1.2}
% 		\small
% 		\centering
% 		\caption{\label{tab:propriedades} Principais propriedades geométricas do duto}
% 		\begin{tabular}{lcr}
% 			\toprule
% 			Característica & Unidade & Valor\\
% 			\midrule
% 			Comprimento do duto &  m & 1000\\
% 			Diâmetro externo &  in & 10,75\\
% 			Espessura nominal &  in & 0,50\\
% 			Espessura do revestimento &  mm & 2,7\\
% 			Densidade do aço & kg/m$^3$ & 7850\\
% 			Densidade do revestimento & kg/m$^3$ & 923\\
% 			Densidade do conteúdo & kg/m$^3$ & 200\\
% 			Módulo de elasticidade & Pa & 2,07 $\times 10^{11}$\\
% 			Limite de escoamento d do aço (API 5L-X60) & Pa & 4,14 $\times 10^8$\\
% 			Comprimento dos elementos (MEF) & m & 0,25\\
% 			\bottomrule
% 		\end{tabular}
% \end{table}

Para iniciar as análises, todos esses dados devem estar devidamente estruturados num arquivo de entrada no formato JSON.
Para carregar esses dados e criar uma instância da classe \texttt{Model}, pode-se usar a função \texttt{load\_json}, como mostrado na \autoref{code:load-json}.


\begin{figure} % TODO: usar pacote correto
	\caption{Código para carregamento dos dados de entrada.}\label{code:load-json}
	\begin{pythoncode}
from integrispan.model_generator.model_generator import load_json
model = load_json("inputs.json")
	\end{pythoncode}
\end{figure}

Antes de iniciar a simulação, que pode ser um processo custoso, é importante revisar se o modelo está minimamente de acordo com o desejado. Pode-se, por exemplo, visualizar o perfil da batimetria tomando partido de uma das funções gráficas implementadas no módulo \texttt{plots}. O código para geração desse gráfico está na \autoref{code:perfil-da-batimetria}, cujo resultado é mostrado na \autoref{fig:ex_perfil_da_batimetria}.

\begin{figure}[!ht] % TODO: usar pacote correto
\caption{Código para geração de \autoref{fig:ex_perfil_da_batimetria} com o perfil da batimetria}\label{code:perfil-da-batimetria}
\begin{pythoncode}
from integrispan.plots import plots
bathymetry_plot = plots.bathymetry(model.bat)
bathymetry_plot.save("perfil_batimetria.html")
\end{pythoncode}
\fonte{Autor (2020)}
\end{figure}

\begin{figure}[!ht]
    \centering
    \caption{Perfil do modelo.}\label{fig:ex_perfil_da_batimetria}
    \includegraphics[width=\textwidth]{imagens/exemplo/ex_perfil_da_batimetriat.png}
    \fonte{Autor (2020)}
\end{figure}

A simulação realizada tem a seguinte sequência de passos de carga:

\begin{enumerate}
	\item Aplica-se o peso do duto vazio;
	\item Aplica-se a pressão externa;
	\item Aplica-se a tração de lançamento;
	\item Aplica-se o deslocamento vertical e assenta-se o duto;
	\item Restaura-se o atrito axial;
	\item Ativa-se as molas;
	\item Remove-se a tração de lançamento;
	\item Aplica-se a pressão do teste hidrostático;
	\item Remove-se a pressão do teste hidrostático;
	\item Aplica-se a pressão operacional;
	\item Obtêm-se os modos de vibração (análise modal).
\end{enumerate}

% Os valores dos carregamentos estão presentes na \autoref{tab:carregamentos}.

% \begin{table}[h]
% 	\renewcommand{\arraystretch}{1.2}
% 	\small
% 	\centering
% 	\caption{\label{tab:carregamentos} Carregamentos do duto.}
% 	\begin{tabular}{lcr}
% 		\toprule
% 		Característica & Unidade & Valor\\
% 		\midrule
% 		Pressão externa & kgf/cm$^2$ & 160\\
% 		Coeficiente de atrito transversal & & 0,9\\
% 		Coeficiente de atrito longitudinal & & 0,6\\
% 		Tração Residual de Lançamento & kN & 60\\
% 		Pressão do teste hidrostático & kgf/cm$^2$ & 160\\ % TODO
% 		Pressão operacional & kgf/cm$^2$ & 160\\ % TODO
% 		\bottomrule
% 	\end{tabular}
% \end{table}

A geração dos arquivos para simulação no ABAQUS contendo as instruções para toda essa sequência de passo é realizada usando o método \texttt{write\_inps}: \texttt{model.write\_inps()}.
Isso deve gerar dois arquivos dentro de um diretório chamado \texttt{exemplo}: o arquivo principal \texttt{exemplo.inp}, e \texttt{bt\_exemplo.inp}, com as coordenadas que define o perfil de batimetria. \par

O método \texttt{run\_abaqus} do objeto \texttt{model} é responsável por executar a chamada do ABAQUS de maneira programática para iniciar a simulação.
Nesse método, ocorre a leitura do arquivo de registro da simulação e o seu conteúdo é exibido na tela do console a cada \unit[5]{s}.

Uma vez terminada a simulação, a extração de dados do arquivo \texttt{odb} gerado pelo ABAQUS e o pós-processamento podem ser realizados com chamada do método \texttt{post\_processing} do objeto \texttt{model}: \texttt{model.post\_processing()}.
Para cada tipo de dados extraído, há uma função para representação gráfica desses resultados. Nas figuras a seguir são apresentados os gráficos para alguns deles. %TODO indicar quais figuras?

Em geral, a primeira forma de validação é uma inspeção visual da configuração deformada do duto sobre a batimetria. Desta forma é possível ver se a simulação consegue reproduzir a situação \textit{in-loco}, especialmente os vãos. Uma vez que a etapa anterior tenha gerado os arquivos \texttt{deformed\_shape.csv} (com as coordenadas da configuração deformada do duto) e \texttt{seabed.csv}, (com as coordenadas da batimetria), pode-se gerar um gráfico com o perfil do duto. O código para geração do gráfico desejado é semelhante ao exposto na \autoref{code:deformada}, e o gráfico resultante na \autoref{fig:ex-config-deformada}. Este e outros gráficos mais comuns são gerados automaticamente com uma chamada do método \texttt{make\_plots} do objeto \texttt{model}: \texttt{model.make\_plots()}.

\begin{figure}[!ht] % TODO: usar pacote correto
\caption{Código para geração do gráfico do perfil da configuração deformada do duto sobre a batimetria.}\label{code:deformada}
\begin{pythoncode}
deformed_shape = load_csv(model.final_results_folder / "deformed_shape.csv")
pipe_plot = plots.pipe_profile(deformed_shape, legend="Eixo do duto")
seabed = load_csv(model.final_results_folder / "seabed.csv")
pipe_plot.title = "Configuração deformada"
seabed_plot = plots.bathymetry(seabed)
seabed_pipe_plot = pipe_plot + seabed_plot
seabed_pipe_plot.set_range(y=(-0.06, 0.06), x=(400, 600))
seabed_pipe_plot.save("configuração_deformada.html")
\end{pythoncode}
\fonte{Autor (2020)}
\end{figure}

Vale destacar o uso do operador de adição (\texttt{+}) na linha 6. Com ele, é possível combinar dois gráficos (instâncias da classe Plot), sobrepondo-os em um mesmo gráficos.

\begin{figure}[!ht]
	\centering
	\caption{Configuração deformada do duto após a simulação.}\label{fig:ex-config-deformada}
	\includegraphics[width=\textwidth]{imagens/exemplo/deformada}
	\fonte{Autor (2020)}
\end{figure}

O próximo passo é a visualização dos modos de vibração com o objetivo de selecionar os que irão ser utilizados para o cálculo de fadiga. As Figuras~\ref{fig:all-modos-IL} e~\ref{fig:all-modos-CF} apresentam todos os modos de vibração computados pela análise modal do ABAQUS, separando-os em \textit{in-line} e \textit{cross-flow}, onde a região que compreende a extensão do vão está destacada com o fundo cinza.

\begin{figure}[H]
	\centering
	\caption{Modos de vibração \textit{in-line} geradas na análise modal.}\label{fig:all-modos-IL}
	\includegraphics[width=\textwidth]{imagens/exemplo/all_modos_IL}
	\fonte{Autor (2020)}
\end{figure}

\begin{figure}[H]
	\centering
	\caption{Modos de vibração \textit{cross-flow} geradas na análise modal.}\label{fig:all-modos-CF}
	\includegraphics[width=\textwidth]{imagens/exemplo/all_modos_CF}
	\fonte{Autor (2020)}
\end{figure}
% TODO Muitos modos espúrios em uma modelagem simples. Por quê? Faltaram informações sobre: dados de entrada, qual tipo/tamanho elementos, a deflexão estática está de acordo com o esperado, satisfez as condições para usar as formulas analíticas da DNVGL? Por que não utilizou dados mais realistas de vão e diâmetro?

Como é possível ver nas Figuras~\ref{fig:all-modos-IL} e~\ref{fig:all-modos-CF}, há muitos modos espúrios, os quais devem ser desconsiderados da análise de fadiga. Esse é o papel do método \texttt{select\_modes} da classe \texttt{Span}. Esse método consiste na implementação dos processos de seleção de modos recomendados pela \dnvf105, apresentados na~\autoref{sec:multimode}. Para o exemplo aqui ilustrado, os resultados são os modos exibidos nas Figuras~\ref{fig:modos-IL} e~\ref{fig:modos-CF}.

\begin{figure}[!ht]
	\centering
	\caption{Modos de vibração \textit{in-line} selecionados pelo algoritmo implementado.}\label{fig:modos-IL}
	\includegraphics[width=\textwidth]{imagens/exemplo/modos_IL}
	\fonte{Autor (2020)}
\end{figure}

\begin{figure}[!ht]
	\centering
	\caption{Modos de vibração \textit{cross-flow} selecionados pelo algoritmo implementado.}\label{fig:modos-CF}
	\includegraphics[width=\textwidth]{imagens/exemplo/modos_CF}
	\fonte{Autor (2020)}
\end{figure}

Estando satisfeito com os modos selecionados na etapa anterior, o usuário pode usar o método \texttt{run\_fatfree}, para que seja feita o preenchimento de uma instância da planilha com os dados específicos do vão, modos, e dados mais gerais, como condições ambientais e coeficientes de segurança. Um exemplo da massiva entrada de dados necessária nesse processo acontece na aba \textit{Multimode} da \fatfree, onde se faz a entrada das coordenadas dos modos de vibração (colunas em azul), conforme é possível ver na \autoref{fig:fatfree-multimode}. % TODO: referencias no texto dessa forma é melhor do dizer que a figura exibe alguma coisa

\begin{figure}[!ht]
	\centering
	\caption{Dados inseridos na aba \textit{Multimode} da FatFree.}\label{fig:fatfree-multimode}
	\includegraphics[width=0.6\textwidth]{imagens/exemplo/fatfree_multimode}
	\fonte{Autor {2020}}
\end{figure}

Com essa instância da planilha, tem-se outro ponto para validação dos resultados.
Segundo o item 6.7.4 da \dnvf105, a análise de elementos finitos para um único vão com $L / D_s \approx 60$, as frequências naturais de \textit{in-line} e \textit{cross-flow} e as faixas de tensão devem mostrar valores semelhantes com margem de $\pm$5\%, considerando esforço axial efetivo nulo ($S_{eff} = 0$). Para o modelo em questão, a comparação entre os valores da primeira frequência obtidos pela análise modal do ABAQUS e a \fatfree\ é apresentada na \autoref{tab:comparacao-frequencias}. % TODO Na comparação utilizou esforço efetivo nulo?

\begin{table}[!ht]
\renewcommand{\arraystretch}{1.2}
\centering
\caption{Comparação entre valores para a primeira frequência.}\label{tab:comparacao-frequencias}
\begin{tabular}{lrrr}
\toprule
Direção & ABAQUS & FatFree & Diferença relativa\\
\midrule
\textit{In-line}    & 1,315 & 1,351 & -2,66\%\\ % chktex 21
\textit{Cross-flow} & 1,317 & 1,357 & -2,95\%\\ % chktex 21
\bottomrule
\end{tabular}
\end{table}

Uma vez que se tenha atingido sucesso na tarefa de modelar o problema de modo a condizentes, pode-se agora usar a classe \texttt{FatFreeResults} para extrair os resultados de uma dada instância da \fatfree. Por exemplo, pode-se construir um gráfico da vida à fadiga ao longo do vão com perfil do duto com a batimetria abaixo dele (ver \autoref{fig:ex-config-deformada}), como mostra a \autoref{code:vida-a-fadiga}.

\begin{figure}[!ht] % TODO: usar pacote correto
	\caption{Código paa geração do gráficos de vida à fadiga.}\label{code:vida-a-fadiga}
	\begin{pythoncode}
from integrispan.dnv.fatfree_results import FatFreeResults
fatfree_results = FatFreeResults("FatFree_kp_508.5.xls", "./")
fatigue_life_plot = fatfree_results.plot_fatigue_life(baseline=20)
dashboard = fatigue_life_plot / seabed_pipe_plot
dashboard.save("fatigue_pipe_dashboard.html")
	\end{pythoncode}
	\fonte{Autor (2020)}
\end{figure}

Vale destacar o uso do operador de divisão (\texttt{/}) na linha 4. Com ele, é possível combinar dois gráficos (instâncias da classe Plot) posicionado-os um sobre o outro em um mesmo gráfico. O gráfico resultante é exibido na \autoref{fig:vida-a-fadiga}, onde se tem a vida à fadiga nas direções \textit{in-line} (IL Comb) e \textit{cross-flow} (CF RM) na região do vão, bem como uma linha de referência que ajuda a identificar os trechos cuja previsão de vida à fadiga está abaixo do esperado.

\begin{figure}[!ht]
	\centering
	\caption{Gráfico de vida à fadiga e perfil do duto.}\label{fig:vida-a-fadiga}
	\includegraphics[width=\textwidth]{imagens/exemplo/vida-a-fadiga}
	\fonte{Autor (2020)}
\end{figure}
% TODO: Na vida real para um vão de 19m não era esperado uma vida a fadiga tão baixa. Poderia ter usado dados de entrada mais realista.

% TODO Da mesma forma que no capítulo anterior, acho que falta um fechamento para este capítulo. Talvez um exposição especifica sobre a qualidade dos resultados alcançados.

% TODO: um dos objetivos da implementação do framework é economia de hh/tempo. qual foi a estimativa de redução?
\chapter{Considerações finais}\label{chap:conclusao}

\section{Sugestões de trabalhos futuros}

% ----------------------------------------------------------
% ELEMENTOS PÓS-TEXTUAIS (Referências, Glossário, Apêndices)
% ----------------------------------------------------------
\postextual

\bibliography{bibliografia}

\apendices
\renewcommand{\apendicesname}{AP\^ENDICES}
\partpage

\chapter{Descrição do formato do arquivo de entrada\label{apendice:json}}

JSON é um formato de arquivo em texto puro que representa informações atribuindo um nome (ou rótulo) que descreve o seu significado e a seguir, o seu valor. Esta sintaxe de representação é derivada da forma utilizada pelo JavaScript para representar objetos. Muito mais que um formato de arquivo, é um modelo para armazenamento e transmissão de informações no formato texto e que é bastante utilizado por aplicações web. A representação de informações utilizada em arquivos JSON é simples, e sua forma de estruturação é bem mais compacta do que a que normalmente é feita em arquivos XML, o que torna o processamento das informações muito mais rápido. A especificação completa do formato JSON pode ser encontrado em www.json.org.

O arquivo foi feito pensando na facilidade do usuário em reconhecer os rótulos e preenchê-los de forma fá\newline
cil e prática.
O arquivo compreende informações que servirão tanto para a análise do ABAQUS (construindo um ou mais arquivos \texttt{.inp}), quanto para a \fatfree.
A seguir descreve-se brevemente a estruturação das chaves no arquivo de entrada, apresentado nas \autoref{lst:model} a \autoref{lst:fatfree}.

\section*{Palavras-chave no arquivo de entrada}

\subsection{\texttt{MODEL}}

Dados gerais do modelo, como:

\begin{itemize}
  \item \texttt{NAME}: Nome do modelo. Usado como base para o nome dos arquivos e diretórios criados;
  \item \texttt{BATIMETRIA}: indicação do arquivo de batimetria;
  \item \texttt{RESULTS\_FOLDER}: indicação do caminho para armazenamento dos resultados;
  \item \texttt{SPRING\_PIPE\_EXTREMITY}: indicação da extremidade onde será colocada a mola, a mesma na qual foi aplicada a tração residual de lançamento.
\end{itemize}

\begin{figure}
\caption{Exemplo de arquivo de entrada de dados: chave \texttt{MODEL}.}\label{lst:model}
\begin{jsoncode}
{
  "MODEL": {
    "NAME": "Duto_Piloto_X",
    "BATIMETRIA": "C:/Batimetria/batimetria.csv",
    "RESULTS_FOLDER": "C:/Resultados",
    "SPRING_PIPE_EXTREMITY": 0
  }
}
\end{jsoncode}
\end{figure}

\subsection{\texttt{FILE\_BAT}}

Parâmetros para a execução da simulação no ABAQUS.

\begin{itemize}
  \item \texttt{CPU}: número de núcleos disponíveis para simulação;
  \item \texttt{GPU}: número de núcleos da placa gráfica disponíveis para simulação;
  \item \texttt{INTERACTIVE}: garante que as simulações ocorram sequencialmente, e não simultaneamente.
\end{itemize}

\begin{figure}
\caption{Exemplo de arquivo de entrada de dados: chave \texttt{FILE\_BAT}.\label{lst:file_bat}}
\begin{jsoncode}
{
  "FILE_BAT": {
    "CPU": 12,
    "GPU": 12,
    "INTERACTIVE": 0
  },
}
\end{jsoncode}
\end{figure}

\subsection{\texttt{CONDITIONS}}

Configurações gerais do modelo:

\begin{itemize}
  \item \texttt{TYPE\_SEABED}:
  \begin{itemize}
    \item \texttt{0}: Batimetria modelada com elementos do tipo R3D4.
    \item \texttt{1}: Batimetria modelada como superfície analítica,
  \end{itemize}
  \item  \texttt{CURTAIN\_SPRINGS}:
    \begin{itemize}
      \item \texttt{0}: A rigidez do solo é prescrita.
      \item \texttt{1}: A rigidez vertical é modelada como molas do tipo \textit{spring}.
    \end{itemize}
  \item \texttt{RUN\_MODEL}:
    \begin{itemize}
      \item \texttt{0}: Utilizar o programa apenas para pós-processamento.
      \item \texttt{1}: Rodar o modelo no ABAQUS.
    \end{itemize}
  \item \texttt{POST\_PROCESSING}:
    \begin{itemize}
      \item \texttt{0}: Rodar apenas a simulação no ABAQUS.
      \item \texttt{1}: Após a simulação, iniciar diretamente o pós processamento.
    \end{itemize}
  \item \texttt{ISIGHT}: chama diretamente o iSight caso seja necessário um processo de otimização ou DOE\@.
  \item \texttt{SUPPORTS}: indica se os suportes devem ser inseridos no modelo.
  \item \texttt{DELETE\_FOLDER}: indica se o conteúdo da pasta de resultados deve ser apagado antes da nova simulação
\end{itemize}

\begin{figure}
\caption{Exemplo de arquivo de entrada de dados: chave \texttt{CONDITIONS}.\label{lst:conditions}}
\begin{jsoncode}
{
  "CONDITIONS": {
    "TYPE_SEABED": 1,
    "CURTAIN_SPRINGS": 0,
    "RUN_MODEL": 1,
    "POST_PROCESSING": 1,
    "ISIGHT": 0,
    "SUPPORTS": true,
    "DELETE_FOLDER": false
  },
}
\end{jsoncode}
\end{figure}

\subsection{\texttt{MODE\_SELECTOR}}

Parâmetros necessários para o módulo de seleção de frequências.

\begin{itemize}
  \item \texttt{NODESET}: indica o \textit{nodeset} referente ao duto;
  \item \texttt{ELEMENTSET}: indica o \textit{elset} referente ao duto;
  \item \texttt{SEABEDSET}: indica o set referente à batimetria;
  \item \texttt{SPANS}: dados dos vãos que serão submetidos ao pós-processamento.
  \item \texttt{N\_MODES\_IN\_LINE}: indica o número de nós que serão extraídos na direção \textit{in-line}.
  \item \texttt{N\_MODES\_CROSS}: indica o número de nós que serão extraídos na direção \textit{cross-flow}.
\end{itemize}

\begin{figure}
\caption{Exemplo de arquivo de entrada de dados: chave \texttt{MODE\_SELECTOR}.\label{lst:mode-selector}}
\begin{jsoncode}
{
  "MODE_SELECTOR": {
    "NODESET": "PIPE",
    "ELEMENTSET": "PIPE",
    "SEABEDSET": "M_FUNDO",
    "SPANS": [
        {
          "kp_inicial": 9958.2,
          "kp_final": 9975.2,
          "azimute": 60.00
        }
      ],
    "N_MODES_IN_LINE": 4,
    "N_MODES_CROSS": 3
  }
}
\end{jsoncode}
\end{figure}


\subsection{\texttt{PIPE\_GEOMETRY}}


Indica as informações relativas ao duto, como comprimento, altura de lançamento e tamanho do elemento.

\begin{itemize}
  \item \texttt{INITIAL\_KP}: ponto de início do duto a ser modelado;
  \item \texttt{END\_KP}: ponto final do duto modelado;
  \item \texttt{LAUNCH\_HEIGHT}: altura de lançamento do duto, sob o qual será modelado a superfície fictícia;
  \item \texttt{LendTH\_ELEMENT}: comprimento do elemento do tipo PIPE31 utilizado na modelagem do duto.
\end{itemize}

\begin{figure}
\caption{Exemplo de arquivo de entrada de dados: chave \texttt{PIPE\_GEOMETRY}.\label{lst:pipe-geometry}}
\begin{jsoncode}
{
  "PIPE_GEOMETRY": {
    "INITIAL_KP": 0,
    "END_KP": 1000,
    "LAUNCH_HEIGHT": -100,
    "LendTH_ELEMENT": 0.25
  },
}
\end{jsoncode}
\end{figure}

\subsection{\texttt{AUXILIARY\_NODE}}

Fornece os dados para a criação de nós auxiliares.

\begin{itemize}
  \item \texttt{OFFSET\_NODE\_SPRING}: define o \textit{offset} para a alocação da mola;
  \item \texttt{INCREASE\_FICTION\_PLAN\_X}: aumento lateral do plano fictício na extremidade inicial;
  \item \texttt{INCREASE\_FICTION\_PLAN\_Y}: aumento lateral do plano fictício na extremidade final;
\end{itemize}

\begin{figure}
\caption{Exemplo de arquivo de entrada de dados: chave \texttt{AUXILIARY\_NODE}.\label{lst:auxiliary-node}}
\begin{jsoncode}
{
  "AUXILIARY_NODE": {
    "OFFSET_NODE_SPRING": 3,
    "INCREASE_FICTION_PLAN_X": 100,
    "INCREASE_FICTION_PLAN_Y": 10
  },
}
\end{jsoncode}
\end{figure}

\subsection{\texttt{CURTAIN\_SPRINGS}}

Caso o modelo seja simulado com cortina de molas, é necessário passar os dados para sua modelagem.

\begin{itemize}
  \item \texttt{HEIGHT}: cota vertical inicial das molas;
  \item \texttt{STIFFNESS}: dados de rigidez das molas.
\end{itemize}

\begin{figure}
\caption{Exemplo de arquivo de entrada de dados: chave \texttt{CURTAIN\_SPRINGS}.\label{lst:curtain-springs}}
\begin{jsoncode}
{
  "CURTAIN_SPRINGS": {
    "HEIGHT": -60,
    "STIFFNESS": [
      [
        -207904,
        -1
      ],
      [
        0,
        0
      ],
      [
        0,
        1
      ]
    ]
  }
}
\end{jsoncode}
\end{figure}


\subsection{\texttt{SPRING\_STIFFNESS}}

Fornece os dados da rigidez das molas nas extremidades do duto.

\begin{itemize}
  \item \texttt{INITIAL}: dados da mola da extremidade inicial;
  \item \texttt{END}: dados da mola da extremidade final.
\end{itemize}

\begin{figure}
\caption{Exemplo de arquivo de entrada de dados: chave \texttt{SPRING\_STIFFNESS}.\label{lst:spring_stiffness}}
\begin{jsoncode}
{
  "SPRING_STIFFNESS": {
    "INITIAL": [
      [
        -1.0001e9,
        -10.0
      ],
      [
        1.0001e9,
        10.0
      ]
    ],
    "END": [
      [
        -1.0001e9,
        -10.0
      ],
      [
        1.0001e9,
        10.0
      ]
    ]
  },
}
\end{jsoncode}
\end{figure}


\subsection{\texttt{PIPE\_MATERIAL}}

Propriedades físicas e geométricas da sessão do duto.

\begin{itemize}
  \item \texttt{DENSITY}: densidade;
  \item \texttt{ELASTICITY\_MODULE}: módulo de elasticidade;
  \item \texttt{POISSON}: coeficiente de \textit{Poisson};
  \item \texttt{COEFFICIENT\_EXPANSION}: coeficiente de expansão térmica;
  \item \texttt{YIELD\_STRESS}: limite de escoamento do aço;
  \item \texttt{PLASTIC\_DEFORMATION}: limite de escoamento;
  \item \texttt{EXTERNAL\_RADIUS}: raio externo do duto;
  \item \texttt{THICKNESS}: espessura da parede duto.
\end{itemize}

\begin{figure}
\caption{Exemplo de arquivo de entrada de dados: chave \texttt{PIPE\_MATERIAL}.\label{lst:pipe_material}}
\begin{jsoncode}
{
  "PIPE_MATERIAL": {
    "DENSITY": 9850.0,
    "ELASTICITY_MODULE": 2.07e11,
    "POISSON": 0.3,
    "COEFFICIENT_EXPANSION": 1.17e-5,
    "YIELD_STRESS": 4.15e8,
    "PLASTIC_DEFORMATION": 0.00,
    "EXTERNAL_RADIUS": 0.200,
    "THICKNESS": 0.0150
  },
}
\end{jsoncode}
\end{figure}


\subsection{CONTACT\_PIPE\_SEABED}


Propriedades do contato entre o duto e a superfície referente à batimetria.

\begin{itemize}
  \item \texttt{HCRIT}: define quanto a superfície \textit{slave} pode penetrar na superfície \textit{master} antes que o ABAQUS abandone o incremento atual e tente novamente com um incremento menor;
  \item \texttt{ELASTIC\_SLIP}: escoamento elástico permissível a ser usado no método de rigidez para aderência à fricção
  \item \texttt{FRICTION\_COEFF\_1}: coeficiente de atrito;
  \item \texttt{FRICTION\_COEFF\_2}: coeficiente de atrito;
  \item \texttt{STABILIZE}: parâmetro de estabilidade do contato;
  \item \texttt{STIFFNESS}: dados de rigidez do solo, para interação duto-solo.
\end{itemize}

\begin{figure}
\caption{Exemplo de arquivo de entrada de dados: chave \texttt{MODE\_SELECTOR}.\label{lst:mode_selector}}
\begin{jsoncode}
{
  "CONTACT_PIPE_SEABED": {
    "HCRIT": 10000,
    "ELASTIC_SLIP": 0.001,
    "FRICTION_COEFF_1": 0.6,
    "FRICTION_COEFF_2": 0.8,
    "STABILIZE": 1e-8,
    "STIFFNESS": [
      [
        0,
        0
      ],
      [
        1000,
        1
      ]
    ]
  },
}
\end{jsoncode}
\end{figure}


\subsection{\texttt{CONTACT\_PIPE\_PLAN}}

Fornece as propriedades do contato entre o duto e a superfície fictícia.

\begin{figure}
\caption{Exemplo de arquivo de entrada de dados: chave \texttt{CONTACT\_PIPE\_PLAN}.\label{lst:contact_pipe_plan}}
\begin{jsoncode}
{
  "CONTACT_PIPE_PLAN": {
    "HCRIT": 10000,
    "ELASTIC_SLIP": 0.001,
    "EXTENSION_ZONE": 0.2,
    "FRICTION_COEFF_1": 0,
    "FRICTION_COEFF_2": 0.8,
    "STABILIZE": 1e-8,
    "STIFFNESS": [
      [
        0,
        0
      ],
      [
        1000,
        0.1
      ]
    ]
  },
}
\end{jsoncode}
\end{figure}


\subsection{\texttt{STEPS\_DEFAULTS}}

Reúne as informações recorrentes em todos os passos de carga (\textit{steps}) no arquivo \texttt{*.inp}.

\begin{itemize}
  \item \texttt{MAXIMUM\_INCREMENT\_NUMBER}: define o máximo de incrementos para o \textit{step};
  \item \texttt{AUTOMATIC\_STABILIZATION}: parâmetro de estabilização para casos onde há contato;
  \item \texttt{INITIAL\_SIZE\_INCREMENT}: valor inicial do incremento;
  \item \texttt{TOTAL\_STEP\_TIME}: tempo total no \textit{step};
  \item \texttt{MINIMUM\_INCREMENT\_SIZE}: número mínimo de incrementos para o \textit{step};
  \item \texttt{MAXIMUM\_INCREMENT\_SIZE}: número máximo de incrementos para o \textit{step}.
\end{itemize}

\texttt{STEP\_X}: informação específica para cada \textit{step} conforme os passos de carga anteriormente citados. Caso existam informações específicas do suporte, essas informações devem ser passadas respectivamente. Caso contrário, são utilizadas as informações do \texttt{STEP\_DEFAULTS}.

\begin{figure}
\caption{Exemplo de arquivo de entrada de dados: chave \texttt{STEPS\_DEFAULTS}.\label{lst:steps_defaults}}
\begin{jsoncode}
{
"STEPS_DEFAULTS": {
  "MAXIMUM_INCREMENT_NUMBER": 10000,
  "AUTOMATIC_STABILIZATION": 1.0e-8,
  "INITIAL_SIZE_INCREMENT": 0.0001,
  "TOTAL_STEP_TIME": 0.01,
  "MINIMUM_INCREMENT_SIZE": 1.0e-30,
  "MAXIMUM_INCREMENT_SIZE": 0.01
  },
  "STEP_1": {
    "EMPTY_SUBMERSE_WEIGHT": 1.4
    },
    "STEP_2": {
      "INITIAL_SIZE_INCREMENT": 0.1,
      "TOTAL_STEP_TIME": 2.1,
      "MAXIMUM_INCREMENT_SIZE": 2.1,
      "EXTERNAL_PRESSURE": 103542.25,
      "EFFECTIVE_DIAMETER": 0.4523
  },
  "STEP_3": {
    "INITIAL_SIZE_INCREMENT": 0.001,
    "RELEASE_TRACTION": 35000.00
  },
  "STEP_4": {
    "MAXIMUM_INCREMENT_NUMBER": 100000,
    "DISPLACEMENT_FICTITIOUS_PLANE": -200.00
  },
  "STEP_5": {},
  "STEP_6": {},
  "STEP_7": {},
  "STEP_8": {
    "INITIAL_SIZE_INCREMENT": 0.1,
  "TOTAL_STEP_TIME": 24.93,
  "MAXIMUM_INCREMENT_SIZE": 5.00,
  "INTERNAL_PRESSURE": 3654453.0,
  "WEIGHT_SUBMERGED_FLOODED": 2.62
  },
  "STEP_9": {
    "INITIAL_SIZE_INCREMENT": 0.1,
    "TOTAL_STEP_TIME": 24.93,
    "MAXIMUM_INCREMENT_SIZE": 5
  },
  "STEP_10": {
    "INITIAL_SIZE_INCREMENT": 0.1,
    "TOTAL_STEP_TIME": 15.12,
    "MAXIMUM_INCREMENT_SIZE": 5,
    "WEIGHT_SUBMERGED_OPERATIONAL": 0.85,
    "OPERATION_PRESSURE": 1265413
  },
  "STEP_11": {
    "NUMBER_MODES": 100,
    "MAXIMUM_NUMBER_INTERACTIONS": 200
  },
}
\end{jsoncode}
\end{figure}

\subsection{\texttt{SUPPORTS}}

Reúne as informações referentes à posição dos suportes.

\begin{itemize}
  \item \texttt{FILE\_DEFORMED\_IN\_LOCO}: caminho para o arquivo de \textit{topping}. O arquivo é utilizado como base de comparação para alocação dos suportes;
  \item \texttt{SHIFT\_SURFACE}: %%%%%%%% ;
  \item \texttt{STEP}: \textit{step} a partir do qual serão alocados os suportes;
  \item \texttt{LIST}: lista de suportes de acordo com o tipo implementado. São passados o KP centro do suporte e o comprimento do mesmo.
  \begin{itemize}
    \item No primeiro conjunto de colchetes são passados suportes do tipo \textit{grout bag}, posicionados abaixo do duto;
    \item Em seguida, suportes do tipo manta, posicionados acima do duto;
    \item No terceiro colchete, suportes mecânicos do tipo pino (restrição de deslocamentos transversais ao eixo do duto e as rotações nodais).
    \item Por fim, suportes mecânicos do tipo livre (restrição de deslocamento transversal ao eixo do duto)
  \end{itemize}
\end{itemize}

\begin{figure}
\caption{Exemplo de arquivo de entrada de dados: chave \texttt{SUPPORTS}.\label{lst:supports}}
\begin{jsoncode}
{
  "SUPPORTS": {
    "FILE_DEFORMED_IN_LOCO": "D:/Batimetria/arquivo_survey.eff",
    "SHIFT_SURFACE": 10,
    "STEP": 7,
    "LIST": [
      [
        [
          845,
          1
        ]
      ],
      [],
      [],
      []
    ]
  },
}
\end{jsoncode}
\end{figure}


\subsection{\texttt{STEP\_NODAL\_FIX}}

\textit{Step} no qual serão aplicadas as restrições nodais, relacionadas aos diferentes tipos de suporte. Chave vazia refere-se aos valores passados na \textit{keyword} \texttt{STEPS\_DEFAULTS}.

\subsection{\texttt{STEP\_SURFACE\_SUPPORTS}}

\textit{Step} no qual as superfícies analíticas referentes aos suportes são posicionadas. Pode-se incluir as chaves presentes na chave \texttt{STEPS\_DEFAULTS} para modificar os valores.


\subsection{\texttt{FATFREE}}

Passa os dados necessários para preenchimento da \fatfree~via \texttt{Xlwings}.

\begin{itemize}
  \item {CURRENT\_FILE}: indica caminho para o arquivo de corrente, retirado das ETs fornecidas pela na forma de histograma. A rotina realiza os cálculos necessários para incluir os dados na \textit{sheet Current} da planilha de cálculo;
  \item \texttt{ISIGHT\_FILE}: indica o caminho onde será salvo o arquivo \texttt{*.txt} com os dados que serão utilizados para o pós-processamento no iSight;
  \item \texttt{Flag}: permite o usuário escolher se os dados de rigidez (Kv, KL e  Kv\_s) serão os passados pelo usuário acima ou extraídos do ABAQUS.
\end{itemize}

\begin{figure}
\caption{Exemplo de arquivo de entrada de dados: chave \texttt{FATFREE}.}\label{lst:fatfree}
\begin{jsoncode}
{
  "FATFREE": {
    "CURRENT_FILE": "D:\\Corrente\\arquivo_de_corrente.csv",
    "ISIGHT_FILE": "D:\\Resultados\\DutoX_trecho3\\isight.txt",
    "h": 165,
    "L": 70,
    "e": 0.88,
    "d": 0,
    "teta_pipe": 0,
    "z_structure": 0.005,
    "z_soil_in_line": 0.02,
    "z_soil_cross_flow": 0.014,
    "z_hRM": 0,
    "Kv": 1.33e7,
    "KL": 1e7,
    "Kv_s": 2.5e5,
    "SCF": 1,
    "kc": 0,
    "fcn": 42,
    "k": 0.0033,
    "p": 124.54,
    "DT": 0.02062,
    "Ds": 0.3238,
    "t_concrete": 0,
    "t_coating": 0.003,
    "r_steel": 7850,
    "r_concrete": 0,
    "r_coating": 935,
    "r_cont": 200,
    "Turbulence_intensity_Ic": 0.04,
    "Measurement_ref_Height_zr": 5,
    "On_bottom_roughness_z0": 0.00004,
    "Time_between_independent_current_events": 1.0,
    "Flag": false
  }
}
\end{jsoncode}
\end{figure}

% \chapter{Arquivo de entrada usado no exemplo de aplicação\label{apendice:arquivo-exemplo}}

% \begin{figure}
% \caption{Conteúdo do arquivo de \texttt{exemplo.json}\label{lst:input-json}}
% \inputminted[frame=single,fontsize=\tiny,linenos]{python}{exemplo/exemplo.json}
% \end{figure}

\end{document}
