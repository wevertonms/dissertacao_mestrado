\apendices\

\chapter{Arquivo de entrada\label{apendice:json}}

JSON é um formato de arquivo em texto puro que representa informações atribuindo um nome (ou rótulo) que descreve o seu significado e a seguir, o seu valor. Esta sintaxe de representação é derivada da forma utilizada pelo JavaScript para representar objetos. Muito mais que um formato de arquivo, é um modelo para armazenamento e transmissão de informações no formato texto e que é bastante utilizado por aplicações web. A representação de informações utilizada em arquivos JSON é muito simples e sua forma de estruturação é bem mais compacta do que a que normalmente é feita em arquivos XML, o que torna o processamento das informações muito mais rápido.

O arquivo foi feito pensando na facilidade do usuário em reconhecer os rótulos e preenchê-los de forma fácil e prática.
O arquivo compreende informações que servirão tanto para a análise do ABAQUS (construindo um ou mais arquivos \texttt{.inp}) quanto para a \fatfree.
A seguir descreve-se brevemente a estruturação das chaves no arquivo de entrada, apresentado nos códigos-fontes na \autoref{lst:model} ao \autoref{lst:fatfree}.

\section{Palavras-chaves e descrições}

\subsection{\texttt{MODEL}}

Dados gerais do modelo, como:

\begin{itemize}
  \item \texttt{NAME}: Nome do arquivo;
  \item \texttt{BATIMETRIA}: indicação do arquivo de batimetria;
  \item \texttt{RESULTS\_FOLDER}: indicação do caminho para armazenamento dos resultados;
  \item \texttt{SPRING\_PIPE\_EXTREMITY}: indicação da extremidade onde será colocada a mola, a mesma na qual foi aplicada a tração residual de lançamento.
\end{itemize}

\captionof{listing}{Exemplo de arquivo de entrada de dados: chave \texttt{MODEL}.\label{lst:model}}
\begin{jsoncode}
{
  "MODEL": {
    "NAME": "Duto_Piloto_X",
    "BATIMETRIA": "C:/Batimetria/batimetria.csv",
    "RESULTS_FOLDER": "C:/Resultados",
    "SPRING_PIPE_EXTREMITY": 0
  }
}
\end{jsoncode}

\subsection{\texttt{FILE\_BAT}}

Parâmetros para a execução da simulação no ABAQUS.

\begin{itemize}
  \item \texttt{CPU}: número de núcleos disponíveis para simulação;
  \item \texttt{GPU}: número de núcleos da placa gráfica disponíveis para simulação;
  \item \texttt{INTERACTIVE}: garante que as simulações ocorram sequencialmente, e não simultaneamente.
\end{itemize}

\captionof{listing}{Exemplo de arquivo de entrada de dados: chave \texttt{FILE\_BAT}.\label{lst:file_bat}}
\begin{jsoncode}
{
  "FILE_BAT": {
    "CPU": 12,
    "GPU": 12,
    "INTERACTIVE": 0
  },
}
\end{jsoncode}

\subsection{\texttt{CONDITIONS}}

Configurações gerais do modelo:

\begin{itemize}
  \item \texttt{TYPE\_SEABED}:
  \begin{itemize}
    \item \texttt{0}: Batimetria modelada com elementos do tipo R3D4.
    \item \texttt{1}: Batimetria modelada como superfície analítica,
  \end{itemize}
  \item  \texttt{CURTAIN\_SPRINGS}:
    \begin{itemize}
      \item \texttt{0}: A rigidez do solo é prescrita.
      \item \texttt{1}: A rigidez vertical é modelada como molas do tipo \textit{spring}.
    \end{itemize}
  \item \texttt{RUN\_MODEL}:
    \begin{itemize}
      \item \texttt{0}: Utilizar o programa apenas para pós-processamento.
      \item \texttt{1}: Rodar o modelo no ABAQUS.
    \end{itemize}
  \item \texttt{POST\_PROCESSING}:
    \begin{itemize}
      \item \texttt{0}: Rodar apenas a simulação no ABAQUS.
      \item \texttt{1}: Após a simulação, iniciar diretamente o pós processamento.
    \end{itemize}
  \item \texttt{ISIGHT}: chama diretamente o iSight caso seja necessário um processo de otimização ou DOE\@.
  \item \texttt{SUPPORTS}: indica se os suportes devem ser inseridos no modelo.
  \item \texttt{DELETE\_FOLDER}: indica se o conteúdo da pasta de resultados deve ser apagado antes da nova simulação
\end{itemize}

\captionof{listing}{Exemplo de arquivo de entrada de dados: chave \texttt{CONDITIONS}.\label{lst:conditions}}
\begin{jsoncode}
{
  "CONDITIONS": {
    "TYPE_SEABED": 1,
    "CURTAIN_SPRINGS": 0,
    "RUN_MODEL": 1,
    "POST_PROCESSING": 1,
    "ISIGHT": 0,
    "SUPPORTS": true,
    "DELETE_FOLDER": false
  },
}
\end{jsoncode}

\subsection{\texttt{MODE\_SELECTOR}}

Parâmetros necessários para o módulo de seleção de frequências.

\begin{itemize}
  \item \texttt{NODESET}: indica o \textit{nodeset} referente ao duto;
  \item \texttt{ELEMENTSET}: indica o \textit{elset} referente ao duto;
  \item \texttt{SEABEDSET}: indica o set referente à batimetria;
  \item \texttt{SPANS}: dados dos vãos que serão submetidos ao pós-processamento.
  \item \texttt{N\_MODES\_IN\_LINE}: indica o número de nós que serão extraídos na direção \textit{in-line}.
  \item \texttt{N\_MODES\_CROSS}: indica o número de nós que serão extraídos na direção \textit{cross-flow}.
\end{itemize}

\captionof{listing}{Exemplo de arquivo de entrada de dados: chave \texttt{MODE\_SELECTOR}.\label{lst:mode-selector}}
\begin{jsoncode}
{
  "MODE_SELECTOR": {
    "NODESET": "PIPE",
    "ELEMENTSET": "PIPE",
    "SEABEDSET": "M_FUNDO",
    "SPANS": [
        {
          "kp_inicial": 9958.2,
          "kp_final": 9975.2,
          "azimute": 60.00
        }
      ],
    "N_MODES_IN_LINE": 4,
    "N_MODES_CROSS": 3
  }
}
\end{jsoncode}

\subsection{\texttt{PIPE\_GEOMETRY}}

Indica as informações relativas ao duto, como comprimento, altura de lançamento e tamanho do elemento.

\begin{itemize}
  \item \texttt{INITIAL\_KP}: ponto de início do duto a ser modelado;
  \item \texttt{END\_KP}: ponto final do duto modelado;
  \item \texttt{LAUNCH\_HEIGHT}: altura de lançamento do duto, sob o qual será modelado a superfície fictícia;
  \item \texttt{LENGTH\_ELEMENT}: comprimento do elemento do tipo PIPE31 utilizado na modelagem do duto.
\end{itemize}

\captionof{listing}{Exemplo de arquivo de entrada de dados: chave \texttt{PIPE\_GEOMETRY}.\label{lst:pipe-geometry}}
\begin{jsoncode}
{
  "PIPE_GEOMETRY": {
    "INITIAL_KP": 0,
    "END_KP": 1000,
    "LAUNCH_HEIGHT": -100,
    "LENGTH_ELEMENT": 0.25
  },
}
\end{jsoncode}

\subsection{\texttt{AUXILIARY\_NODE}}

Fornece os dados para a criação de nós auxiliares.

\begin{itemize}
  \item \texttt{OFFSET\_NODE\_SPRING}: define o \textit{offset} para a alocação da mola;
  \item \texttt{INCREASE\_FICTION\_PLAN\_X}: aumento lateral do plano fictício na extremidade inicial;
  \item \texttt{INCREASE\_FICTION\_PLAN\_Y}: aumento lateral do plano fictício na extremidade final;
\end{itemize}

\captionof{listing}{Exemplo de arquivo de entrada de dados: chave \texttt{AUXILIARY\_NODE}.\label{lst:auxiliary-node}}
\begin{jsoncode}
{
  "AUXILIARY_NODE": {
    "OFFSET_NODE_SPRING": 3,
    "INCREASE_FICTION_PLAN_X": 100,
    "INCREASE_FICTION_PLAN_Y": 10
  },
}
\end{jsoncode}

\subsection{\texttt{CURTAIN\_SPRINGS}}

Caso o modelo seja simulado com cortina de molas, é necessário passar os dados para sua modelagem.

\begin{itemize}
  \item \texttt{HEIGHT}: cota vertical inicial das molas;
  \item \texttt{STIFFNESS}: dados de rigidez das molas.
\end{itemize}

\captionof{listing}{Exemplo de arquivo de entrada de dados: chave \texttt{CURTAIN\_SPRINGS}.\label{lst:curtain-springs}}
\begin{jsoncode}
{
  "CURTAIN_SPRINGS": {
    "HEIGHT": -60,
    "STIFFNESS": [
      [
        -207904,
        -1
      ],
      [
        0,
        0
      ],
      [
        0,
        1
      ]
    ]
  }
}
\end{jsoncode}

\begin{jsoncode}
{
  "SPRING_STIFFNESS": {
    "INITIAL": [
      [
        -1.0001e9,
        -10.0
      ],
      [
        1.0001e9,
        10.0
      ]
    ],
    "END": [
      [
        -1.0001e9,
        -10.0
      ],
      [
        1.0001e9,
        10.0
      ]
    ]
  },
}
\end{jsoncode}

\subsection{\texttt{PIPE\_MATERIAL}}

Propriedades físicas e geométricas da sessão do duto.

\begin{itemize}
  \item \texttt{DENSITY}: densidade;
  \item \texttt{ELASTICITY\_MODULE}: módulo de elasticidade;
  \item \texttt{POISSON}: coeficiente de \textit{Poisson};
  \item \texttt{COEFFICIENT\_EXPANSION}: coeficiente de expansão térmica;
  \item \texttt{YIELD\_STRESS}: limite de escoamento do aço;
  \item \texttt{PLASTIC\_DEFORMATION}: limite de escoamento;
  \item \texttt{EXTERNAL\_RADIUS}: raio externo do duto;
  \item \texttt{THICKNESS}: espessura da parede duto.
\end{itemize}

\captionof{listing}{Exemplo de arquivo de entrada de dados: chave \texttt{PIPE\_MATERIAL}.\label{lst:pipe_material}}
\begin{jsoncode}
{
  "PIPE_MATERIAL": {
    "DENSITY": 9850.0,
    "ELASTICITY_MODULE": 2.07e11,
    "POISSON": 0.3,
    "COEFFICIENT_EXPANSION": 1.17e-5,
    "YIELD_STRESS": 4.15e8,
    "PLASTIC_DEFORMATION": 0.00,
    "EXTERNAL_RADIUS": 0.200,
    "THICKNESS": 0.0150
  },
}
\end{jsoncode}

\subsection{\texttt{SPRING\_STIFFNESS}}

Fornece os dados da rigidez das molas nas extremidades do duto.

\begin{itemize}
  \item \texttt{INITIAL}: dados da mola da extremidade inicial;
  \item \texttt{END}: dados da mola da extremidade final.
\end{itemize}


\subsection{CONTACT\_PIPE\_SEABED}


Propriedades do contato entre o duto e a superfície referente à batimetria.

\begin{itemize}
  \item \texttt{HCRIT}: define quanto a superfície \textit{slave} pode penetrar na superfície \textit{master} antes que o ABAQUS abandone o incremento atual e tente novamente com um incremento menor;
  \item \texttt{ELASTIC\_SLIP}: escoamento elástico permissível a ser usado no método de rigidez para aderência à fricção
  \item \texttt{FRICTION\_COEFF\_1}: coeficiente de atrito;
  \item \texttt{FRICTION\_COEFF\_2}: coeficiente de atrito;
  \item \texttt{STABILIZE}: parâmetro de estabilidade do contato;
  \item \texttt{STIFFNESS}: dados de rigidez do solo, para interação duto-solo.
\end{itemize}

\captionof{listing}{Exemplo de arquivo de entrada de dados: chave \texttt{MODE\_SELECTOR}.\label{lst:mode_selector}}
\begin{jsoncode}
{
  "CONTACT_PIPE_SEABED": {
    "HCRIT": 10000,
    "ELASTIC_SLIP": 0.001,
    "FRICTION_COEFF_1": 0.6,
    "FRICTION_COEFF_2": 0.8,
    "STABILIZE": 1e-8,
    "STIFFNESS": [
      [
        0,
        0
      ],
      [
        1000,
        1
      ]
    ]
  },
}
\end{jsoncode}


\subsection{\texttt{CONTACT\_PIPE\_PLAN}}

Fornece as propriedades do contato entre o duto e a superfície fictícia.

\begin{jsoncode}
{
  "CONTACT_PIPE_PLAN": {
    "HCRIT": 10000,
    "ELASTIC_SLIP": 0.001,
    "EXTENSION_ZONE": 0.2,
    "FRICTION_COEFF_1": 0,
    "FRICTION_COEFF_2": 0.8,
    "STABILIZE": 1e-8,
    "STIFFNESS": [
      [
        0,
        0
      ],
      [
        1000,
        0.1
      ]
    ]
  },
}
\end{jsoncode}


\subsection{STEPS\_DEFAULTS}

Reúne as informações recorrentes em todos os passos de carga (\textit{steps}) no arquivo \texttt{*.inp}.

\begin{itemize}
  \item \texttt{MAXIMUM\_INCREMENT\_NUMBER}: define o máximo de incrementos para o \textit{step};
  \item \texttt{AUTOMATIC\_STABILIZATION}: parâmetro de estabilização para casos onde há contato;
  \item \texttt{INITIAL\_SIZE\_INCREMENT}: valor inicial do incremento;
  \item \texttt{TOTAL\_STEP\_TIME}: tempo total no \textit{step};
  \item \texttt{MINIMUM\_INCREMENT\_SIZE}: número mínimo de incrementos para o \textit{step};
  \item \texttt{MAXIMUM\_INCREMENT\_SIZE}: número máximo de incrementos para o \textit{step}.
\end{itemize}

\texttt{STEP\_X}: informação específica para cada \textit{step} conforme os passos de carga anteriormente citados. Caso existam informações específicas do suporte, essas informações devem ser passadas respectivamente. Caso contrário, são utilizadas as informações do \texttt{STEP\_DEFAULTS}.

\captionof{listing}{Exemplo de arquivo de entrada de dados: chave \texttt{STEPS\_DEFAULTS}.\label{lst:steps_defaults}}
\begin{jsoncode}
{
"STEPS_DEFAULTS": {
  "MAXIMUM_INCREMENT_NUMBER": 10000,
  "AUTOMATIC_STABILIZATION": 1.0e-8,
  "INITIAL_SIZE_INCREMENT": 0.0001,
  "TOTAL_STEP_TIME": 0.01,
  "MINIMUM_INCREMENT_SIZE": 1.0e-30,
  "MAXIMUM_INCREMENT_SIZE": 0.01
  },
  "STEP_1": {
    "EMPTY_SUBMERSE_WEIGHT": 1.4
    },
    "STEP_2": {
      "INITIAL_SIZE_INCREMENT": 0.1,
      "TOTAL_STEP_TIME": 2.1,
      "MAXIMUM_INCREMENT_SIZE": 2.1,
      "EXTERNAL_PRESSURE": 103542.25,
      "EFFECTIVE_DIAMETER": 0.4523
  },
  "STEP_3": {
    "INITIAL_SIZE_INCREMENT": 0.001,
    "RELEASE_TRACTION": 35000.00
  },
  "STEP_4": {
    "MAXIMUM_INCREMENT_NUMBER": 100000,
    "DISPLACEMENT_FICTITIOUS_PLANE": -200.00
  },
  "STEP_5": {},
  "STEP_6": {},
  "STEP_7": {},
  "STEP_8": {
    "INITIAL_SIZE_INCREMENT": 0.1,
  "TOTAL_STEP_TIME": 24.93,
  "MAXIMUM_INCREMENT_SIZE": 5.00,
  "INTERNAL_PRESSURE": 3654453.0,
  "WEIGHT_SUBMERGED_FLOODED": 2.62
  },
  "STEP_9": {
    "INITIAL_SIZE_INCREMENT": 0.1,
    "TOTAL_STEP_TIME": 24.93,
    "MAXIMUM_INCREMENT_SIZE": 5
  },
  "STEP_10": {
    "INITIAL_SIZE_INCREMENT": 0.1,
    "TOTAL_STEP_TIME": 15.12,
    "MAXIMUM_INCREMENT_SIZE": 5,
    "WEIGHT_SUBMERGED_OPERATIONAL": 0.85,
    "OPERATION_PRESSURE": 1265413
  },
  "STEP_11": {
    "NUMBER_MODES": 100,
    "MAXIMUM_NUMBER_INTERACTIONS": 200
  },
}
\end{jsoncode}

\subsection{\texttt{SUPPORTS}}

Reúne as informações referentes à posição dos suportes.

\begin{itemize}
  \item \texttt{FILE\_DEFORMED\_IN\_LOCO}: caminho para o arquivo de \textit{topping}. O arquivo é utilizado como base de comparação para alocação dos suportes;
  \item \texttt{SHIFT\_SURFACE}: %%%%%%%% ;
  \item \texttt{STEP}: \textit{step} a partir do qual serão alocados os suportes;
  \item \texttt{LIST}: lista de suportes de acordo com o tipo implementado. São passados o KP centro do suporte e o comprimento do mesmo.
  \begin{itemize}
    \item No primeiro conjunto de colchetes são passados suportes do tipo \textit{grout bag}, posicionados abaixo do duto;
    \item Em seguida, suportes do tipo manta, posicionados acima do duto;
    \item No terceiro colchete, suportes mecânicos do tipo pino (restrição de deslocamentos transversais ao eixo do duto e as rotações nodais).
    \item Por fim, suportes mecânicos do tipo livre (restrição de deslocamento transversal ao eixo do duto)
  \end{itemize}
\end{itemize}

\captionof{listing}{Exemplo de arquivo de entrada de dados: chave \texttt{SUPPORTS}.\label{lst:supports}}
\begin{jsoncode}
{
  "SUPPORTS": {
    "FILE_DEFORMED_IN_LOCO": "D:/Batimetria/arquivo_survey.eff",
    "SHIFT_SURFACE": 10,
    "STEP": 7,
    "LIST": [
      [
        [
          845,
          1
        ]
      ],
      [],
      [],
      []
    ]
  },
}
\end{jsoncode}


\subsection{\texttt{STEP\_NODAL\_FIX}}

\textit{Step} no qual serão aplicadas as restrições nodais, relacionadas aos diferentes tipos de suporte. Chave vazia refere-se aos valores passados na \textit{keyword} \texttt{STEPS\_DEFAULTS}.

\subsection{\texttt{STEP\_SURFACE\_SUPPORTS}}

\textit{Step} no qual as superfícies analíticas referentes aos suportes são posicionadas. Pode-se incluir as chaves presentes na chave \texttt{STEPS\_DEFAULTS} para modificar os valores.


\subsection{\texttt{FATFREE}}

Passa os dados necessários para preenchimento da \fatfree~via \texttt{Xlwings}.

\begin{itemize}
  \item {CURRENT\_FILE}: indica caminho para o arquivo de corrente, retirado das ETs fornecidas pela na forma de histograma. A rotina realiza os cálculos necessários para incluir os dados na \textit{sheet Current} da planilha de cálculo;
  \item \texttt{ISIGHT\_FILE}: indica o caminho onde será salvo o arquivo \texttt{*.txt} com os dados que serão utilizados para o pós-processamento no iSight;
  \item \texttt{Flag}: permite o usuário escolher se os dados de rigidez (Kv, KL e  Kv\_s) serão os passados pelo usuário acima ou extraídos do ABAQUS.
\end{itemize}

\captionof{listing}{Exemplo de arquivo de entrada de dados: chave \texttt{FATFREE}.\label{lst:fatfree}}
\begin{jsoncode}
{
  "FATFREE": {
    "CURRENT_FILE": "D:\\Corrente\\arquivo_de_corrente.csv",
    "ISIGHT_FILE": "D:\\Resultados\\DutoX_trecho3\\isight.txt",
    "h": 165,
    "L": 70,
    "e": 0.88,
    "d": 0,
    "teta_pipe": 0,
    "z_structure": 0.005,
    "z_soil_in_line": 0.02,
    "z_soil_cross_flow": 0.014,
    "z_hRM": 0,
    "Kv": 1.33e7,
    "KL": 1e7,
    "Kv_s": 2.5e5,
    "SCF": 1,
    "kc": 0,
    "fcn": 42,
    "k": 0.0033,
    "p": 124.54,
    "DT": 0.02062,
    "Ds": 0.3238,
    "t_concrete": 0,
    "t_coating": 0.003,
    "r_steel": 7850,
    "r_concrete": 0,
    "r_coating": 935,
    "r_cont": 200,
    "Turbulence_intensity_Ic": 0.04,
    "Measurement_ref_Height_zr": 5,
    "On_bottom_roughness_z0": 0.00004,
    "Time_between_independent_current_events": 1.0,
    "Flag": false
  }
}
\end{jsoncode}