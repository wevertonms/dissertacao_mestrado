\chapter{Desenvolvimento do \frame}\label{chap:software}

O levantamento dos requisitos de um sistema é o elemento que fornece elementos que deve nortear uma série de decisões a serem tomadas no seu desenvolvimento. Primeiramente, apresenta-se aqui o fluxo de trabalho tradicional para análise de fadiga.

\section{Fluxo de avaliação de vida à fadiga em dutos em vão livre}\label{chap:workflow}


Baseado nos estudos e \textit{oficinas} realizados para o desenvolvimento deste trabalho. Pôde-se estabelecer que a análise de vida a fadiga em dutos em vão livre compreende o fluxograma apresentado na \autoref{fig:fluxograma}.

\begin{figure}[!ht]
    \centering
    \caption{Fluxo de avaliação de vida à fadiga em dutos em vão livre.}\label{fig:fluxograma}
    \includegraphics[width=\textwidth]{imagens/fluxograma.pdf}
    \fonte{Autor (2020)}
\end{figure}

A seguir, uma breve descrição de cada item:

\begin{enumerate}[label=(\arabic*)]
    \item Nesta etapa, o profissional reúne as informações básicas para construção dos modelos e outros dados usados em cálculos posteriores. Citadamente, temos aqui: os as cotas do perfil do duto e batimetria obtidas na inspeção, geometria e composição das camadas que compõem sessão do duto, parâmetros do solo, constantes físicas e coeficientes de segurança, posição e tipos de suportes ao longo do duto. Essa tarefa envole olhar uma série de documentos (\texttt{.doc}, \texttt{.pdf}, etc) em busca desses valores, dispostos de forma não estruturada. Quando estruturados, em forma de arquivos CSV ou planilhas, por exemplo, é necessário ainda manipular esses dados a fim de extrair somente a informação necessária ou convertê-las no formato apropriado. Um exemplo disso são os dados de batimetria, que precisam convertidos nas coordenadas dos nós de uma malha de elementos finitos.
    \item 
\end{enumerate}{}


\section{Fluxo de avaliação de vida à fadiga com uso do \frame}


O \frame\ proposto tem como requisito atender o fluxo apresentado na \autoref{sec:workflow}, automatizando certas etapas da análise de vida à fadiga. Os itens coloridos são cobertos pela implementação \frame, e os itens  branco são realizados em aplicações externas.


\begin{figure}[!ht]
    \centering
    \caption{Fluxo de operação proposto para o \frame.}\label{fig:workflow}
    \includegraphics[width=\textwidth]{imagens/fluxograma_automatizado}
\end{figure}

De forma mais detalhada, a ferramente deve:

\begin{enumerate}[label= (\arabic*)]
    \item A partir de um arquivo de entrada com informações do modelo, criar arquivos de entrada para o ABAQUS (.inp) que reproduza todo o processo de simulação do comportamento do duto apresentado (\autoref{sec:assentamento}).
    \item Submeter o arquivo gerado para análise no ABAQUS. No caso de haver a colocação de suportes, a simulação é executada em duas partes: a primeira com os passos de carga anteriores a passo da colocação dos suportes, e a segunda com o passo da colação dos mesmos e os passos de carga seguintes.
    \item Processar os arquivos de saída do ABAQUS (.odb) extraindo as informações relevantes como a configuração deformada, modos de vibração, etc., gerando arquivos em outros formatos de fácil leitura para pós-processamento, tanto por este \frame, quanto por outros softwares.
    \item Pós-processar as informações gerando gráficos e relatórios relevantes para as tomadas de decisão do usuário quanto ao projeto. Esse é o requisito mais crítico, uma vez que é fundamental o entendimento sobre a análise de duto em vão-livre. Entre as tarefas que fazem parte deste item está a automação da escolha dos modos de vibração ativos e relevantes e para cada vão de interesse --- a qual deve ser norteada pelos aspectos discutidos na \autoref{sec:multimode} --- e a manipulação da FatFree.
    \item Ativar o processo de cálculo de fadiga no arquivo preenchido no passo anterior.
    \item Capturar os resultados no arquivo \texttt{.xls} da \fatfree, que agora contém os resultados do cálculo de vida à fadiga, e apresentá-los na forma de gráficos e relatórios.
\end{enumerate}


\section{Implementação computacional}


\subsection{Linguagem de programação}

Python\footnote{https://www.python.org} foi a linguagem de programação adotada. Além de ser uma linguagem interpretada de alto nível Orientada a Objeto --- que permite um alto índice de reaproveitamento de código --- e da sintaxe simples. \citeonline{Rao2018} apresenta algumas das principais vantagens que destaca a linguagem para este tipo de aplicação:

\begin{itemize}
    \item Disponibilidade de bibliotecas para aplicações científicas contemplando manipulação de matrizes (Numpy), funções matemáticas (SciPy), manipulação de dados em forma tabular (Pandas), criação de gráficos interativos (Matplotlib e Bokeh).

    \item Suporte para automação de tarefas. Os recursos de \textit{script} internos do Python e vários pacotes têm um forte suporte à automação de tarefas. A automação de tarefas repetitivas e a realização do registro de dados são fáceis e requerem pouco esforço. O ABAQUS, por exemplo, permite modelagem e acesso a informações em arquivos de saída via Python. A biblioteca xlwings permite manipulação de planilhas Excel, a exemplo da FatFree.

    \item Pacotes Python como Django e Flask tornam possível desenvolver e usar o Python como uma API\footnote{Na programação de computadores, uma Interface de Programação de Aplicativos (\textit{Application Programming Interface}---API) é um conjunto de definições de sub-rotinas e ferramentas para a criação de software. Em termos gerais, é um conjunto de métodos de comunicação claramente definidos entre vários componentes.} com um \textit{front-end} da web. Essa funcionalidade é particularmente útil para reaproveitamento do \frame\  em outras aplicações.
\end{itemize}


\subsection{Pacotes e classes}


Para implementação do fluxo de trabalho proposto para o \frame, fez-se a implementação de módulos para lidar com cada contexto específico. Em python, módulos podem ser quaisquer arquivos com extensão \texttt{.py}. Estes módulos podem ser agrupados em pacotes, que são pastas que, além dos módulos, contém um arquivo \texttt{\_\_init\_\_.py}. No \frame\ têm-se alguns pacotes que agrupam módulos em torno do contexto de um problema que o \textit{software} resolve. % chktex 21


\subsubsection{Pacote \texttt{analysis}}


É o pacote principal responsável orquestrar o fluxo de trabalho do \frame\ desde o processamento dos dados de entrada, geração dos arquivos para o ABAQUS e os pós-processamentos. As funções de pré-processamento de dados estão aqui. Este pacote tem dois módulos (\texttt{models.py} e \texttt{inp.py}) que contém duas classes principais, respectivamente:

\begin{itemize}
    \item \texttt{Model}: classe que contém as informações do modelo do problema.
    A classe armazena todas as informações para construção dos arquivos \texttt{.inp}, isto é, dados de batimetria, material, geometria do duto, coeficientes de segurança, entre outros.
    A principal forma de criação da instâncias dessa classe é pelo método estático \texttt{load\_json}, que recebe um arquivo principal de entrada (em formato JSON, ver \autoref{apendice:json}), e realiza o pré-processamento dos dados contidos nele.

    \item \texttt{Inp}: lida com a escrita modularizada de arquivos de entrada  para o ABAQUS. A proposta é que se tenha um arquivo \texttt{.inp} principal que terá conterá informações para a localização de outros arquivos acessórios que, por sua vez, terão as informações específicas de cada aspecto da modelagem: batimetria, passos de carga, etc. Isso facilita o reaproveitamento dos arquivos acessórios, sem precisa de alterações no arquivo principal.
\end{itemize}

A \autoref{fig:analysis-uml} exibe um diagrama UML\footnote{Sigla para \textit{Unified Modeling Language}: Linguagem Unificada de Modelagem é uma linguagem padrão para modelagem orientada a objetos~\cite{infoescolauml}} com uma visão geral do pacote \texttt{analysis}.

\begin{figure}[!ht]
    \centering
    \caption{Digrama UML do pacote \texttt{analysis}.}\label{fig:analysis-uml}
    \includegraphics[width=0.8\textwidth]{imagens/analysis-uml}
    \fonte{Autor (2020)}
\end{figure}


\subsubsection{Pacote \texttt{odb\_handler}}


Neste pacote está o módulo responsável por lidar com os arquivos de saída do ABAQUS (odb) no sentido de acessar e guardar os dados relevantes em arquivos com formatos de fácil manipulação por outros softwares (CSV e JSON, por exemplo, que possui módulos para leitura e escrita nativos em Python).

% A \autoref{fig:odb-handler-uml} exibe um diagrama UML com uma visão geral do pacote \texttt{odb\_handler}.

\begin{figure}[!ht]
    \centering
    \caption{Digrama UML do pacote \texttt{odb\_handler}.}\label{fig:odb-handler-uml}
    \includegraphics[width=0.5\textwidth]{imagens/odb-handler-uml}
    \fonte{Autor (2020)}
\end{figure}


\subsubsection{Pacote \texttt{mode\_selector}}


Neste pacote estão implementados os métodos responsáveis pela estratégia de seleção automática de modos de vibração (que depende da definição dos vãos), bem como e manipulação dos dados associados vãos e seus respectivos modos. As abstrações dos conceitos de vão e modo de vibração estão implementadas nas seguintes classes:

\begin{itemize}
    \item \texttt{Span}: classe que representa um vão do duto. Uma vez que análise de fadiga é feita por vão, é nesta que são implementados os métodos responsáveis pela seleção dos modos de vibração, que são ligados à classe por um dos seus atributos.

    \item \texttt{ModeShape}: classe que representa um modo de vibração (\textit{mode shape}). A principal forma de criação de objetos dessa classe é por meio da função \texttt{load\_json} do módulo \texttt{mode\_shapes.py}, que carrega os dados do arquivo gerado com utilização do \texttt{odb\_handler}, e retorna uma lista de objetos desta classe.
\end{itemize}

A \autoref{fig:mode-selector-uml} exibe um diagrama UML com uma visão geral do pacote \texttt{mode\_selector}.

\begin{figure}[!ht]
    \centering
    \caption{Digrama UML do pacote \texttt{mode\_selector}.}\label{fig:mode-selector-uml}
    \includegraphics[width=0.8\textwidth]{imagens/mode-selector-uml}
    \fonte{Autor (2020)}
\end{figure}


\subsubsection{Pacote \texttt{dnv}}

Neste pacote são agrupadas módulos referentes à \dnvf105, como cálculos dos modelos de resposta da \dnvf105, subsidiam o algorítmo de seleção de modos de vibração, e entrada de saída de dados da planilha FatFree. As principais classes contidas nesses pacotes são:

\begin{itemize}
    \item \texttt{ResponseModel}: implementa formulação dos modelos de resposta apresentados na \autoref{sec:viv}.

    \item \texttt{FatFree}: responsável fazer a entrada dos dados na planilha e invocar a execução dos cálculos. Essa classe faz uso da biblioteca \texttt{Xlwings}, que consegue se comunicar com o Microsoft Excel e manipular os componentes (caixas de seleção, botões, etc) que formam a interface da FatFree.

    \item \texttt{FatFreeResults}: Esta classe permite o acesso facilita o acesso programático aos dados contidos em um determinado aqui da FatFree.
\end{itemize}

A \autoref{fig:dnv-uml} exibe um diagrama UML com uma visão geral do pacote \texttt{dnv}.

\begin{figure}[!ht]
    \centering
    \caption{Digrama UML do pacote \texttt{dnv}.}\label{fig:dnv-uml}
    \includegraphics[width=\textwidth]{imagens/dnv-uml}
    \fonte{Autor (2020)}
\end{figure}


\subsubsection{Pacote \texttt{plots}}


Pacote responsável pro agregar as funções de geração de gráficos dos resultados. Disponibilizar essas funções dentro do \frame permite que os gráficos com os resultados das análises sejam gerados automaticamente, o visual e forma de apresentação sejam padronizados. As principais classes neste pacote são:

\begin{itemize}

    \item \texttt{Plot}: é construída em cima da classe \texttt{Figure} da biblioteca \texttt{Bokeh}\footnote{www.bokeh.org}, que permite a criação de gráficos interativos em HMTL. Os gráficos gerados por essa biblioteca possuem controles permitem, por exemplo, que o usuário dê zoom ou translate os dados exibidos, ative/desative os ítem mostrado por meio de clique nos respectivos itens nas legendas, sabe o gráfico em formato de imagem estática (\texttt{png}). A \autoref{fig:exemplo-bokeh} exibe um exemplo desses gráficos destacando os elementos interativos em vermelho.

    \begin{figure}[!ht]
        \centering
        \caption{Exemplo de gráfico customizado criado com a biblioteca Bokeh.}\label{fig:exemplo-bokeh}
        \includegraphics[width=0.8\textwidth]{imagens/exemplo-bokeh}
        \fonte{Autor (2020)}
    \end{figure}

    O trabalho da classe \texttt{Plot} é simplificar a criação dos gráficos devido a implementação de métodos que facilitam desde alterar os componentes mais comuns das figuras, como títulos dos eixos e legendas, até combinar gráficos e formar \textit{dashboards} com gráficos integrados. A combinação de gráficos pode ser feita por meio dos operadores matemáticos:
    \begin{itemize}
        \item \texttt{a + b}: para sobrepor os gráficos \texttt{a} e \texttt{b};
        \item \texttt{a | b}: para posicionar dois gráficos, com \texttt{a} à esquerda de \texttt{b};
        \item \texttt{a / b}: para posicionar dois gráficos, com \texttt{a} acima de \texttt{b};
    \end{itemize}

    \item \texttt{PipeProfileDashboards}: esta classe se utilizada da classe \texttt{Plot} para facilitar a criação de um tipo \textit{dashboard} recorrente no processo de análise: dois gráficos empilhados verticalmente, um com dados que podem ser associados a posição ao longo do duto (como esforço axial, modos de vibração, enterramento, vida à fadiga, etc), e o outro abaixo dele com o perfil do duto. Nesse tipo de arranjo é interessante que os eixos das abcissas estejam atrelados, permitindo ao usuário movimentar os dois eixos em sincronia.
\end{itemize}

A \autoref{fig:plots-uml} exibe um diagrama UML com uma visão geral do pacote \texttt{plots}.

\begin{figure}[!ht]
    \centering
    \caption{Digrama UML do pacote \texttt{plots}.}\label{fig:plots-uml}
    \includegraphics[width=0.7\textwidth]{imagens/plots-uml}
    \fonte{Autor (2020)}
\end{figure}