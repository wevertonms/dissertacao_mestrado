\section{Organização do Trabalho}

Nesta seção são apresentados, de forma resumida, os assuntos que serão tratados com mais detalhes em cada capítulo do presente trabalho. No capítulo introdutório apresenta-se o contexto no qual o problema está inserido, bem como algumas informações relevantes para reforçar a importância do tema em estudo. Além disso, são definidos alguns conceitos iniciais a fim de garantir uma melhor compreensão do que está sendo tratado.

O capítulo \ref{chap:fundamentacao} apresenta os conceitos básicos para compressão do problema de determinação da vida à fadiga em dutos em vão livre. Este capítulo está dividido em duas sessões. A primeira (\autoref{sec:assentamento}) apresenta a modelagem computacional do comportamento de dutos submarinos, os passos de carga, e os tipos de elementos empregados para modelar os elementos necessários a simulação. Já a \autoref{sec:viv}, apresenta os conceitos e as formulações por trás da Vibração Induzida por Vórtice (VIV), e as recomendações dados pela referência técnica \dnvf105.

O \autoref{chap:software} é apresentado o fluxo  os aspectos da implementação computacional do \textit{framework}, bem como os princípios norteadores de algumas escolhas, como a escolha da linguagem e paradigma de programação, a estrutura proposta para os módulos e classe.

No \autoref{chap:aplicacoes} tem-se a apresentação do uso da aplicação do \frame\ em uma análise de vida a fadiga para um caso particular. Neste capítulo é possível observar como se faz o uso das principais funções implementadas e a forma de apresentação dos resultados.

No \autoref{chap:conclusao} são tratadas as considerações finais do trabalho, bem como são apresentadas algumas sugestões para trabalhos futuros.