\chapter{Metodologia}\label{chap:metodologia}


Uma vez que o trabalho pretende desenvolver uma ferramente que facilite o fluxo de trabalho do profissional que avalia a vida a fadiga de dutos em vão live, foi necessário, como etapas inciais do trabalho realizar:

\begin{itemize}
    \item Revisão da literatura acerca dos temas nos quais esse profissional lida nesta tarefa - modelagem do assentamento de dutos e vibração induzida por vórtice - a fim de nivelar os conhecimentos básicos necessário para o próximo item.
    \item Entender o completo fluxo de trabalho de profissional, desde a acesso as informações até chegar nos resultados de fadiga. Nessa etapa, é que serão identificados os principais gargalos do processos e aqueles de maior potencial de automatização.
\end{itemize}

Inicialmente a revisão da literatura foi feita por meio de leitura de livros, artigos e referências/recomendações técnicas. Aqui se destacaram a \dnvf105, no tocante ao VIV, e o livro de \citeonline{Bai2014}, na parte de modelagem. Ambos os textos são referências mundiais no assunto, como milhares de citações em outro trabalhos, sendo a \dnvf105 a principal referência no assunto que se propõe.

A etapa de análise do fluxo de trabalho, e parte da etapa de revisão da literatura, foram realizadas por meio de reuniões e oficinas ministrados pelos próprios profissionais no decorrer do desenvolvimento do projeto \integrispan. Onde cada encontro era uma oportunidades de diagnosticar novos itens a serem trabalhados, definindo assim os requisitos da ferramenta a ser desenvolvida.

Superada esta fase de revisão da literatura e algumas oficinas, deu-se inicio ao desenvolvimento
de uma série de \textit{scripts} em linguagem Pyhton (\ref{subsec:python})

\section{Computação científica}

Embora o uso e desenvolvimento de softwares de cunho científico seja uma atividade extremamente importante para pesquisadores, nem sempre são observadas boas práticas são observadas no seu desenvolvimento~\cite{Hannay2009}.
\citeonline{Wilson2014} apresenta um conjunto de boas práticas a serem adotadas no desenvolvimento desse tipo de. A seguir é apresentado o resumo do autor sobre essas práticas:

\vspace{0.5cm}
\begin{tcolorbox}[breakable, enhanced]

\textbf{Boas práticas para computação científica}

\begin{enumerate}
\item \textbf{Escreva programas para pessoas, não para computadores.}
    \begin{enumerate}
        \item Um programa não deve exigir que seus leitores mantenham mais de um punhado de fatos na memória de uma só vez.
        \item Torne os nomes consistentes, distintos e significativos.
        \item Tornar consistente o estilo e a formatação do código.
    \end{enumerate}

\item \textbf{Deixe o computador fazer o trabalho.}
    \begin{enumerate}
        \item Faça o computador repetir tarefas.
        \item Salve comandos recentes em um arquivo para reutilização.
        \item Use uma ferramenta de construção para automatizar fluxos de trabalho.
    \end{enumerate}

\item \textbf{Faça alterações incrementais.}
    \begin{enumerate}
        \item Trabalhe em pequenos passos com feedback frequente e correção de rumo.
        \item Use um sistema de controle de versão.
        \item Coloque tudo o que foi criado manualmente no controle de versão.
    \end{enumerate}

\item \textbf{Não se repita (ou repita outros).}
    \begin{enumerate}
        \item Todos os dados devem ter uma única representação oficial no sistema.
        \item Modularize o código em vez de copiar e colar.
        \item Reutilize o código em vez de reescrevê-lo.
    \end{enumerate}

\item \textbf{Planeje erros.}
    \begin{enumerate}
        \item Adicione asserções aos programas para verificar seu funcionamento.
        \item Use uma biblioteca de testes unitários pronta para uso.
        \item Transforme erros em casos de teste.
        \item Use um depurador simbólico.
    \end{enumerate}

\item \textbf{Otimize o software somente depois que ele funcionar corretamente.}
    \begin{enumerate}
        \item Use um \textit{profiler} para identificar gargalos.
        \item Escreva o código na linguagem de nível mais alto possível.
    \end{enumerate}

\item \textbf{Documente design e finalidade, não a mecânica.}
    \begin{enumerate}
        \item Documente interfaces e razões, não implementações.
        \item Refatore o código, em vez de explicar como ele funciona.
        \item Incorpore a documentação do software no próprio software.
    \end{enumerate}

\item \textbf{Colabore.}
    \begin{enumerate}
        \item Use revisões de código de pré-\textit{merge}.
        \item Use a programação em pares ao interar alguém novo e ao lidar com problemas particularmente difíceis.
        \item Use uma ferramenta de acompanhamento de problemas.
    \end{enumerate}
\end{enumerate}
\end{tcolorbox}


\section{Implementação}


Em observância a estas práticas, algumas decisões relativas ao desenvolvimento da ferramenta foram tomadas, e serão apresentadas a seguir.


\subsection{Linguagem de programação}\label{subsec:python}

A linguagem adotada será Python\footnote{https://www.python.org}. Além de ser uma linguagem interpretada de alto nível Orientada a Objeto -- que permite um alto índice de reaproveitamento de código -- e da sintaxe simples, \citeonline{rao2018} apresenta algumas das principais vantagens que destaca a linguagem para este tipo de aplicação:

\begin{itemize}
    \item Disponibilidade de bibliotecas para aplicações cientificas contemplando manipulação de matrizes (Numpy), funções matemáticas (SciPy), manipulação de dados em forma tabular (Pandas), criação de gráficos interativos (Matplotlib e Bokeh);

    \item Suporte para automação de tarefas. Os recursos de \textit{script} internos do Python e vários pacotes têm um forte suporte à automação de tarefas. A automação de tarefas repetitivas e a realização do registro de dados são fáceis e requerem pouco esforço.

    \item Pacotes Python como Django e Flask tornam possível desenvolver e usar o Python como uma API\footnote{Na programação de computadores, uma Interface de Programação de Aplicativos (\textit{Application Programming Interface} -- API) é um conjunto de definições de sub-rotinas e ferramentas para a criação de software. Em termos gerais, é um conjunto de métodos de comunicação claramente definidos entre vários componentes.} com um \textit{front-end} da web. Essa funcionalidade é particularmente útil ao usar uma infraestrutura baseada em nuvem como plataforma para acessar \textit{back-ends} de computação de alto desempenho (HPC).
\end{itemize}
