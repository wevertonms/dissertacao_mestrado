\chapter{Metodologia}\label{chap:metodologia}


Uma vez que o trabalho pretende desenvolver uma ferramente que facilite o fluxo de trabalho do profissional que avalia a vida a fadiga de dutos em vão live, foi necessário, como etapas inciais do trabalho, realizar:

\begin{itemize}
    \item Revisar a literatura acerca dos temas nos quais esse profissional lida nesta tarefa -- modelagem do assentamento de dutos e vibração induzida por vórtice -- a fim de nivelar os conhecimentos básicos necessário para o próximo item.
    \item Entender o fluxo básico de trabalho de profissional, desde a acesso as informações até chegar nos resultados de fadiga. Nessa etapa, é que foram identificados os principais gargalos do processo, destacando aqueles de maior potencial de automatização.
    \item Inteirar-se das funcionalidades e formas de manipulação das ferramentas usadas nas tarefas de análise, como o \abaqus e a planilha \fatfree.
\end{itemize}

Inicialmente a revisão da literatura foi feita por meio de leitura de livros, artigos e referências/recomendações técnicas.
Aqui se destacaram a \dnvf105, no tocante ao VIV, e o livro de \citeonline{Bai2014}, na parte de modelagem.
Ambos os textos são referências mundiais no assunto, como milhares de citações em outro trabalhos, sendo a \dnvf105 a principal referência no assunto que se propõe.

A etapa de análise do fluxo de trabalho, e parte da etapa de revisão da literatura, foram realizadas por meio de reuniões e oficinas ministrados pelos próprios profissionais no decorrer do desenvolvimento do projeto \integrispan.
Cada encontro foi uma oportunidade de diagnosticar novos itens a serem trabalhados, definindo assim os requisitos da ferramenta a ser desenvolvida, e chegando ao modelo de fluxo de trabalho apresentado mais adiante.

Superada esta fase de revisão da literatura e algumas oficinas, houve o desenvolvimento de uma série de \textit{scripts} em linguagem Python para automação de algumas tarefas.
Essa fase, de caráter exploratório, permitiu o desenvolvimento de pequenas ferramentas que podiam ter feedback mais rápido, melhorando o entendimento dos requisitos e compressão da visão geral do fluxo de trabalho.
Essa forma de trabalho, com \textit{loops} de desenvolvimento com feedbacks rápidos, influenciou fortemente a na adoção de uma linguagem dinâmica para desenvolvimento de \textit{scripts} (ver \autoref{sec:software}).

Tendo em mãos um conjunto inicial de \textit{scripts}, fez-se então a modelagem dos módulos da aplicação com base nas diferentes funcionalidades previstas para a ferramenta.
Os códigos dos \textit{scripts} foram, então, reagrupados nesses módulos, e que se comunicam por meio da utilização das diversas classes e respectivos métodos.
Paralelamente a esta etapa, deu-se a elaboração da especificação do arquivo de entrada para facilitar a passagem de informações a aplicação. 

Finalmente, fez-se a validação da ferramenta usando dados de dutos reais da PETROBRAS.
No entanto, devido a confidencialidade destes dados, neste trabalho não serão apresentados os casos usados a validação, mas casos com dados fictícios para exemplificar as funcionalidades da ferramenta.