\chapter{Metodologia}\label{chap:metodologia}


Uma vez que o trabalho pretende desenvolver uma ferramente que facilite o fluxo de trabalho do profissional que avalia a vida a fadiga de dutos em vão live, foi necessário, como etapas inciais do trabalho realizar:

\begin{itemize}
    \item Revisão da literatura acerca dos temas nos quais esse profissional lida nesta tarefa - modelagem do assentamento de dutos e vibração induzida por vórtice - a fim de nivelar os conhecimentos básicos necessário para o próximo item.
    \item Entender o completo fluxo de trabalho de profissional, desde a acesso as informações até chegar nos resultados de fadiga. Nessa etapa, é que serão identificados os principais gargalos do processos e aqueles de maior potencial de automatização.
\end{itemize}

Inicialmente a revisão da literatura foi feita por meio de leitura de livros, artigos e referências/recomendações técnicas.
Aqui se destacaram a \dnvf105, no tocante ao VIV, e o livro de \citeonline{Bai2014}, na parte de modelagem.
Ambos os textos são referências mundiais no assunto, como milhares de citações em outro trabalhos, sendo a \dnvf105 a principal referência no assunto que se propõe.

A etapa de análise do fluxo de trabalho, e parte da etapa de revisão da literatura, foram realizadas por meio de reuniões e oficinas ministrados pelos próprios profissionais no decorrer do desenvolvimento do projeto \integrispan.
Onde cada encontro era uma oportunidades de diagnosticar novos itens a serem trabalhados, definindo assim os requisitos da ferramenta a ser desenvolvida.

Superada esta fase de revisão da literatura e algumas oficinas, houve o desenvolvimento de uma série de \textit{scripts} em linguagem Pyhton para automação de algumas tarefas.
Essa fase, de caráter exploratório, permitiu o desenvolvimento de pequenas ferramentas que podiam ter feedback mais rápido, melhorando o entendimento dos requisitos e compressão da visão geral do fluxo de trabalho.

Essa forma de trabalho, com loops de feedback rápidos, influenciou fortemente a na adoção de Python como a linguagem de programação adotada.
A linguagem adotada para Implementação da ferramenta será Python\footnote{https://www.python.org}.
Além de ser uma linguagem interpretada de alto nível Orientada a Objeto -- que permite um alto índice de reaproveitamento de código -- e da sintaxe simples, \citeonline{rao2018} apresenta algumas das principais vantagens que destaca a linguagem para este tipo de aplicação:


\begin{itemize}
    \item Disponibilidade de bibliotecas para aplicações cientificas contemplando manipulação de matrizes (Numpy), funções matemáticas (SciPy), manipulação de dados em forma tabular (Pandas), criação de gráficos interativos (Matplotlib e Bokeh);

    \item Suporte para automação de tarefas. Os recursos de \textit{script} internos do Python e vários pacotes têm um forte suporte à automação de tarefas. A automação de tarefas repetitivas e a realização do registro de dados são fáceis e requerem pouco esforço.

    \item Pacotes Python como Django e Flask tornam possível desenvolver e usar o Python como uma API\footnote{Na programação de computadores, uma Interface de Programação de Aplicativos (\textit{Application Programming Interface} -- API) é um conjunto de definições de sub-rotinas e ferramentas para a criação de software. Em termos gerais, é um conjunto de métodos de comunicação claramente definidos entre vários componentes.} com um \textit{front-end} da web. Essa funcionalidade é particularmente útil ao usar uma infraestrutura baseada em nuvem como plataforma para acessar \textit{back-ends} de computação de alto desempenho (HPC).
\end{itemize}
