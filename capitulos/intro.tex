\chapter{Introdução}


Ao longo das últimas décadas, à medida que novos campos de petróleo e gás foram descobertos em águas profundas e distantes da costa, surgiu a necessidade de aplicação de sistemas de coleta e exportação submarinos utilizando dutos rígidos cada vez mais extensos.
Com uma maior extensão, aumentou-se propensão à ocorrência de vãos livres devido à irregularidades do piso marinho.

A configuração de dutos no fundo do mar depende das características topográficas do leito marinho, APAGAR solo, tensão residual, rigidez do tubo e seu peso submerso.
Seções de tubo não suportadas pelo do fundo do mar são chamadas de vãos.
Se o fundo do mar for muito irregular, os dutos tendem a formar vãos em vez de seguir as características topográficas do leito marinho.
Os vãos podem ser criados devido a irregularidades no fundo do mar durante a instalação ou aos subsequentes movimentos horizontais de \textit{scouring} de dutos durante a operação.

A presença de vãos nos dutos no fundo do mar exige uma avaliação para determinar se é necessária uma ação corretiva para evitar danos aos mesmos.
As características estáticas e dinâmicas dos vãos nos dutos devem ser investigadas para garantir que o duto possa ser mantida em um estado aceitavelmente seguro.
Se a segurança necessária não puder ser garantida, as ações corretivas na forma de reencaminhamento, correção de vãos, supressão do VIV e similares são usadas para garantir que os critérios de projeto relativos aos níveis de tensão e possíveis danos à fadiga devido ao VIV não sejam excedidos.

Ainda na fase de projeto, uma avaliação do perfil do fundo do mar ao longo da rota proposta pode ser realizada para identificar se é esperado que ocorram vãos nos dutos.
Se ocorrerem vãos de dutos, será necessário determinar a posições e dimensões de cada um deles.
Esta análise fornece previsões dos números e tamanhos dos vãos esperados.
Devido à quantidade de fatores envolvidos, o processo de lançamento de dutos é uma das tarefas mais desafiadoras, mesmo quando a rota ideal já está definida.
A análise requer o uso de métodos numéricos robustos para seu tratamento.
O Método dos Elementos Finitos (MEF) é amplamente usado nessa tarefa.

De modo a representar adequadamente as condições de campo, é necessário modelar desde a etapa de instalação até a operação do duto, assim como considerar efeito de carregamentos dos diferentes valores de pressões internas e externas nas respectivas etapas.
Modelar a instalação de dutos em um \textit{software} de elementos finitos para uso geral pode ser um trabalho demorado e tedioso, principalmente devido a grandes quantidades de dados da batimetria.
Na maioria das vezes, são necessárias técnicas avançadas de \textit{script} para definir o perfil do leito marinho, selecionar a rota ideal do duto e simular o processo de assentamento~\cite{VandenAbeele2013}.

Nesse cenário, são essenciais ferramentas que auxiliem na pré e pós-processamento de dados e na automação de procedimentos.
Uma ferramenta com essas características traz ganhos significativos para a produtividade e reduzem a probabilidade de erro humano.
Além disso, quando essa ferramenta integra ferramentas de uso específicos (análises e visualização), pode-se reduzir atritos do fluxo de trabalhar que usa várias ferramentas de forma isolada.

\section{Objetivos}

Este trabalho tem como objetivo geral desenvolver uma ferramenta para a análise de fadiga em dutos submarinos em vãos livre, que permita um fluxo de trabalho que inclua de um software de análise de elementos finitos e uma planilha de cálculo de vida a fadiga.
Além disso, este trabalho tem como objetivos específicos:

\begin{itemize}
    \item Modelar e implementar a ferramenta utilizando o paradigma da programação orientada a objetos, através da linguagem Python.
    \item Contribuir para a metodologia de análise de fadiga em dutos por meio da criação de uma metodologia de seleção de modos de vibração.
\end{itemize}