\chapter{Introdução}


Ao longo das últimas décadas, à medida que novos campos de petróleo e gás foram descobertos em águas profundas e distantes da costa, surgiu a necessidade de utilização de sistemas submarinos de coleta e exportação utilizando dutos rígidos cada vez mais extensos.
Com uma maior extensão, aumentou-se a ocorrência de seções de duto que não ficam apoiadas sob o leito marinho devido as irregularidades do solo. Essas seções dos dutos são denominadas trechos em vão-livre. Estas irregularidades podem ser preexistentes durante a instalação ou devido a subsequentes movimentos horizontais de \textit{scouring}\footnote{retirada de solo que suporta o duto devido às intensas correntes de fundo.} de dutos durante a operação.

A presença de trechos dos dutos em vão-livre exige uma avaliação para determinação da necessidade de ações corretivas para evitar danos aos mesmos.
Ainda na fase de projeto, uma avaliação do perfil do fundo do mar ao longo da rota proposta pode ser realizada para identificar se é esperado que haja trechos do duto em vão-livre.
Na existência de tais trechos, é necessária uma análise numérica que forneça previsões dos números e tamanhos dos vãos esperados, que são indicadores da necessidade de possíveis alterações na rota ou ações corretivas.

Devido aos elevados custos ambientais, financeiros, e à imagem da empresa, associada a possíveis acidentes, o transporte seguro de hidrocarbonetos e outros fluidos nos oleodutos é uma das prioridades da indústria de petróleo e gás.
A vibração livre é uma grande preocupação na análise de fadiga de componentes de dutos submarinos, incluindo dutos em vãos-livres~\cite{Gamino2013}.

Neste contexto, o comportamento estático e dinâmico do duto deve ser investigado para garantir a segurança, combatendo o dano estrutural por fadiga, mantendo-o em um estado aceitavelmente seguro.
Se as condições necessárias à segurança não puderem ser garantidas, as ações corretivas na forma de mudança de rota, correção de vãos, supressão do VIV e similares são usadas para garantir que os critérios de projeto relativos aos níveis de tensão e possíveis danos por fadiga devido ao VIV não sejam excedidos.
Para a definição mais assertivas e precisas, uma modelagem estrutural pode ser utilizada, e o Método dos Elementos Finitos (MEF) é amplamente usado nessa tarefa.
A configuração de dutos no fundo do mar depende das características topográficas do leito marinho, características do solo, tensão residual de lançamento, rigidez do duto e seu peso submerso.

Para que as condições de contorno e características do problema simulado reproduzam o comportamento real, é necessário modelar desde a etapa de instalação até a operação do duto, assim como considerar efeito de carregamentos dos diferentes valores de pressões internas e externas nas respectivas etapas.
Modelar a instalação de dutos em um software de elementos finitos para uso geral pode ser um trabalho demorado e tedioso, principalmente devido a grandes quantidades de dados de batimetria.
Na maioria das vezes, é necessário até o uso de \textit{script} na definição do perfil do leito marinho, para poder simular o processo de assentamento~\cite{VandenAbeele2013}.


\section{Fluxo de avaliação de vida à fadiga em dutos em vão livre}\label{chap:workflow}


Baseado nos estudos e \textit{oficinas} realizados para o desenvolvimento deste trabalho. Pôde-se estabelecer que a análise de vida a fadiga em dutos em vão livre compreende o fluxograma apresentado na \autoref{fig:fluxograma}.

\begin{figure}[!ht]
    \centering
    \caption{Fluxo de avaliação de vida à fadiga em dutos em vão livre.}\label{fig:fluxograma}
    \includegraphics[width=\textwidth]{imagens/fluxograma.pdf}
    \fonte{Autor (2020)}
\end{figure}

A seguir, uma breve descrição de cada item:

\begin{enumerate}[label=(\arabic*)]
    \item Nesta etapa, o profissional reúne as informações básicas para construção dos modelos e outros dados usados em cálculos posteriores. Citadamente, temos aqui: os as cotas do perfil do duto e batimetria obtidas na inspeção, geometria e composição das camadas que compõem sessão do duto, parâmetros do solo, constantes físicas e coeficientes de segurança, posição e tipos de suportes ao longo do duto. Essa tarefa envole olhar uma série de documentos (\texttt{.doc}, \texttt{.pdf}, etc) em busca desses valores, dispostos de forma não estruturada. Quando estruturados, em forma de arquivos CSV ou planilhas, por exemplo, é necessário ainda manipular esses dados a fim de extrair somente a informação necessária ou convertê-las no formato apropriado. Um exemplo disso são os dados de batimetria, que precisam convertidos nas coordenadas dos nós de uma malha de elementos finitos.
    \item 
\end{enumerate}{}


\section{Motivação}


No cenário mundial existe a tendência da indústria de óleo e gás de investimento em transformação digital em todas as áreas da cadeia, com desenvolvimento de práticas e ferramentas. Esse movimento levou ao surgimento de ferramentas específicas ao auxílio do profissional responsável pela análise, visualização, predição dos resultados de VIV em dutos em vão-livre~\cite{Mittal2017}. No entanto, pela especificidade dessas ferramentas, seu número ainda é reduzido, destacando-se apenas duas de nível comercial.


% \subsubsection{SAGE Profile}


Desenvolvido pela empresa Fugro, que atua no monitoramento de dutos submarinos, o SAGE Profile~\cite{sageprofile} é um software disponível no mercado para análise global de dutos submarinos.
Por ser específico para este uso, essa aplicação representa avanços em relação à modelagem com softwares de elementos finitos genéricos.
No entanto, ela se limita a análise estrutural, deixando a análise de fadiga sob responsabilidade do usuário.
Além disso, a interação com o usuário é limitada a interface gráfica (GUI\footnote{\textit{Graphical User Interface}: interface de usuário gráfica}, como na \autoref{fig:sageprofile}), o que dificulta a automação de tarefas corriqueiras.

\begin{figure}[!ht]
    \centering
    \caption{Interface gráfica do SAGE Profile.}\label{fig:sageprofile}
    \begin{subfigure}[t]{0.49\textwidth}
        \centering
        \includegraphics[width=\textwidth]{imagens/sage_profile_1}
    \end{subfigure}
    \hfill
    \begin{subfigure}[t]{0.49\textwidth}
        \centering
        \includegraphics[width=\textwidth]{imagens/sage_profile_2}
    \end{subfigure}
    \fonte{www.sage-profile.com.}
\end{figure}


% \subsubsection{Sesam for pipelines}


Em outra frente, tem-se a empresa DNG-GL, que é uma referência mundial, entre outras áreas, na análise de dutos em vão-livre.
Essa empresa é responsável pelo desenvolvimento da suíte Sesam for pipelines~\cite{dnvsesam} focados na análise de dutos submarinos.
Essa suíte consiste em 6 aplicações em VBA\footnote{Uma linguagem pela qual se pode customizar e estender aplicações \textit{desktop} da suíte Microsoft.}, dentre as quais a mais destacada é a \fatfree\footnote{Planilha Microsoft Office Excel desenvolvida pela \dnvf105 focada no cálculo de vida à fadiga de dutos submarinos.} (ver \autoref{fig:fatfree}), responsável pelo cálculo da vida à fadiga em si. Sendo desenvolvidas pela DNG-GL, as aplicações seguem as normas e recomendações práticas propostas pela mesma --- que traz bastante confiabilidade nos resultados. Entretanto, apesar de conter uma aplicação para análise de comportamento mecânico, essas ferramentas são simples e independentes, e estão muito aquém de uma solução completa para simulação estrutural com interação solo-duto, como o SAGE Profile.

\begin{figure}[!ht]
    \centering
    \caption{Interface gráfica (planilha) da \fatfree.}\label{fig:fatfree}
    \includegraphics[width=0.8\textwidth]{imagens/fatfree}
    \fonte{Autor (2020)}
\end{figure}

Pelo exposto sobre o SAGE Profile e o Sesam for pipelines, nota-se que mesmo entre as melhores ferramentas atuais para analisar fadiga de duto em vão-livre ainda há espaço para melhorias, especialmente no que se refere à integração com outras ferramentas que auxiliam e outras etapas do processo. Diante dessas limitações, é comum usar um software para a análise de elementos finitos genérico (como ABAQUS~\cite{Dassault2018} e ANSYS\footnote{https://www.ansys.com}) e realizar o cálculo de fadiga em folhas de cálculo pessoais ou comerciais, como a \fatfree.


Neste cenário, são essenciais ferramentas que possam auxiliar não só no processamento (como os softwares de simulação), mas também o pré e pós-processamento de dados e, até mesmo, na automação de procedimentos.
Uma ferramenta com essas características traz ganhos significativos para a produtividade e reduz a possibilidade de ocorrência de erros humanos.
Além de aumento de produtividade, e redução de falhas associadas a erro humano, tem as questões relacionadas a difícil mobilização e custos dessas operações. Esses itens, caso sejam executados de forma inadequada, podem até amplificar o problema.
Adicionalmente, uma ferramenta que integre softwares de uso específico (para análise e visualização, por exemplo), pode reduzir atritos e padronizar o fluxo de trabalho, em comparação ao uso isolado destes softwares.



\section{Objetivos}


Este trabalho tem como objetivo geral desenvolver uma ferramenta para a análise de fadiga em dutos submarinos em vãos-livres, integrando um software de análise de elementos finitos (ABAQUS) e uma planilha de cálculo de vida à fadiga (\fatfree) em um fluxo de trabalho.

Além disso, este trabalho tem como objetivos específicos:

\begin{itemize}
    \item Contribuir para a metodologia de análise de fadiga em dutos por meio da criação de uma metodologia de seleção de modos de vibração;
    \item Modelar e implementar uma ferramenta utilizando o paradigma da programação orientada a objetos, através da linguagem \textit{Python};
    \item Apresentar a aplicação de algumas das funcionalidades da ferramenta em um estudo de caso.
\end{itemize}


\section{Delimitação do trabalho}


Este trabalho descreve o processo de desenvolvimento de uma ferramenta computacional para auxílio no processo de análise de vida à fadiga de dutos submarinos em vão-livre. A entrega de valor da ferramenta reside na automação de tarefas das etapas de pré e pós-processamento dos dados de entrada do ABAQUS e \fatfree, bem como a possibilidade de execução integrada dessas ferramentas. Portanto, a ferramenta desenvolvida não pretende implementar o processo de análise numérica de elementos finitos, nem o cálculo de vida à fadiga, estes são responsabilidade de ferramentas específicas com esta finalidade, como, por exemplo, ABAQUS e FatFree.


\section{Organização do Trabalho}

O \autoref{chap:viv}: apresenta os conceitos e as formulação por trás da Vibração Induzida por Vórtice (VIV), e as recomendações dados pela referência técnica \dnvf105. O \autoref{chap:assentamento}: apresenta a modelagem computacional do comportamento de duto submarinos, os passos de carga, e os tipos de elementos empregados para modelar os elementos necessários a simulação. O \autoref{chap:software}: apresenta os aspectos da implementação computacional do \textit{framework} , bem como os princípios norteadores de algumas escolhas da, a escolha da linguagem e paradigma de programação, a estrutura proposta para os módulos e classe.