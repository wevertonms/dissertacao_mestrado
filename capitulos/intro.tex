\chapter{Introdução}


Recentemente, as indústrias \textit{offshore} e submarina experimentaram uma revolução técnica no processo de projeto.
Métodos avançados e ferramentas de análise permitiram uma abordagem mais sofisticada ao projeto, que aproveita os materiais modernos e os códigos de projeto revisados, que dão suporte aos conceitos de estado limite de projeto e aos métodos de confiabilidade.
A nova abordagem é chamada de Projeto Através de Análise (\textit{Design Through Analysis} -- DTA), onde o Método dos Elementos Finitos é usado para simular o comportamento global dos dutos, bem como os esforços estruturais locais.
% TODO falta referencias aqui

O processo de duas etapas é usado de maneira complementar para determinar os estados limites normativos e otimizar um projeto específico.

Devido a quantidade de fatores envolvidos, a análise requer o uso de métodos numéricos robustos para seu tratamento
O Método dos Elementos Finitos (MEF) é amplamente usado nessa tarefa.
De modo a representar adequadamente as condições de campo, é necessário modelar desde a etapa de instalação até a operação do duto, assim como considerar efeito de carregamentos dos diferentes valores de pressões internas e externas nas respectivas etapas.

\section{Objetivos}
