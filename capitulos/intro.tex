\chapter{Introdução}


Ao longo das últimas décadas, à medida que novos campos de petróleo e gás foram descobertos em águas profundas e distantes da costa, surgiu a necessidade de utilização de sistemas de coleta e exportação submarinos utilizando dutos rígidos cada vez mais extensos.
Com uma maior extensão, aumentou-se a propensão à ocorrência de vãos livres devido à irregularidades do leito marinho, sejam elas pré-existentes durante a instalação ou devido a subsequentes movimentos horizontais de \textit{scouring} de dutos durante a operação.

A configuração de dutos no fundo do mar depende das características topográficas do leito marinho, solo, tensão residual, rigidez do duto e seu peso submerso.
Seções de duto não suportadas pelo fundo do mar são chamadas de vãos livres.
Se o fundo do mar for muito irregular, os dutos tendem a formar vãos em vez de seguir as características topográficas do leito marinho.

A presença de trechos do duto em vão livre exige uma avaliação para determinar a necessidade de ações corretivas para evitar danos aos mesmos. Ainda na fase de projeto, uma avaliação do perfil do fundo do mar ao longo da rota proposta pode ser realizada para identificar se é esperado que haja trechos do duto em vão livre.
Na existência de tais trechos, será necessária uma análise que forneça previsões dos números e tamanhos dos vãos esperados, que são indicadores da necessidade de possíveis alterações na rota ou ações corretivas.

Devido aos elevados custos (ambientais, financeiros, e à imagem da empresa), associados aos acidentes, o transporte seguro de hidrocarbonetos e outros fluidos nos oleodutos é uma das principais prioridades da indústria de petróleo e gás. A vibração livre é uma grande preocupação na análise de fadiga de componentes de dutos submarinos, incluindo dutos em vãos livres \cite{Gamino2013}.


Sendo assim, o comportamento estático e dinâmico do duto deve ser investigado para garantir a segurança, combatendo o dano estrutural por fadiga, mantendo-o em um estado aceitavelmente seguro.
Se as condições necessárias à segurança não puderem ser garantidas, as ações corretivas na forma de mudança de rota, correção de vãos, supressão do VIV e similares são usadas para garantir que os critérios de projeto relativos aos níveis de tensão e possíveis danos por fadiga devido ao VIV não sejam excedidos.
No entanto, devido à quantidade de fatores envolvidos, o projeto do processo de lançamento de dutos é uma das tarefas mais desafiadoras, mesmo quando a rota ideal já está definida.
A análise requer o uso de métodos numéricos robustos para seu tratamento, e o Método dos Elementos Finitos (MEF) é amplamente usado nessa tarefa.

Para que as condições de contorno e características do problema simulado reproduzam comportamento in-loco, é necessário modelar desde a etapa de instalação até a operação do duto, assim como considerar efeito de carregamentos dos diferentes valores de pressões internas e externas nas respectivas etapas.
Modelar a instalação de dutos em um \textit{software} de elementos finitos para uso geral pode ser um trabalho demorado e tedioso, principalmente devido a grandes quantidades de dados da batimetria.
Na maioria das vezes, são necessárias técnicas avançadas de \textit{script} para definir o perfil do leito marinho, selecionar a rota ideal do duto e simular o processo de assentamento~\cite{VandenAbeele2013}.

Nesse cenário, são essenciais ferramentas que auxiliem no pré e pós-processamento de dados e na automação de procedimentos.
Uma ferramenta com essas características traz ganhos significativos para a produtividade e reduzem a probabilidade de erro humano.
Além disso, uma ferramenta que integre \textit{softwares} de uso específico (para análise e visualização, por exemplo), pode reduzir atritos do fluxo de trabalho, em comparação ao uso isolado destes \textit{softwares}.

Atualmente, existem diversos sistemas submarinos em operação nas Bacias de Campos e Espírito Santo que estão no final ou já ultrapassaram a metade de sua vida útil de projeto, o que torna ainda mais relevante uma ferramenta auxiliar para a reavaliação de integridade e extensão de vida operacional com critérios de cálculo validados.

Por outro lado, no cenário mundial existe a tendência da indústria de óleo e gas de abraçar a transformação digital em todas as áreas da cadeia, com desenvolvimento de práticas e ferramentas para análise, visualização, predição e resultados \cite{Mittal2017}.

\section{Objetivos}

Este trabalho tem como objetivo geral desenvolver uma ferramenta para a análise de fadiga em dutos submarinos em vãos livres,
% que permita um fluxo de trabalho que inclua de um software de análise de elementos finitos e uma planilha de cálculo de vida a fadiga.
que integre um \textit{software} de análise de elementos finitos e o estudo de vida a fadiga.
Além disso, este trabalho tem como objetivos específicos:

\begin{itemize}
    \item Modelar e implementar a ferramenta utilizando o paradigma da programação orientada a objetos, através da linguagem \textit{Python};
    \item Contribuir para a metodologia de análise de fadiga em dutos por meio da criação de uma metodologia de seleção de modos de vibração;
    \item Criar um ambiente interativo de visualização de resultados da simulação com ferramentas que garantam maior liberdade e eficiência para o projetista;
    \item Validar a ferramenta com aplicação de casos.
\end{itemize}

\section{Organização do Trabalho}

O \autoref{chap:viv}: apresenta os conceitos e as formulação por trás da Vibração Induzida por Vórtice (VIV), e as recomendações dados pela referência técnica \dnvf105. O \autoref{chap:assentamento}: apresenta a modelagem computacional do comportamento de duto submarinos, os passos de carga, e os tipos de elementos empregados para modelar os elementos necessários a simulação. O \autoref{chap:software}: apresenta os aspectos da implementação computacional da ferramenta, bem como os princípios norteadores de algumas escolhas da, a escolha da linguagem e paradigma de programação, a estrutura proposta para os módulos e classe.
