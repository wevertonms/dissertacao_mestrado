\chapter{Introdução}


Ao longo das últimas décadas, à medida que novos campos de petróleo e gás foram descobertos em águas profundas e distantes da costa, surgiu a necessidade de utilização de sistemas de coleta e exportação submarinos utilizando dutos rígidos cada vez mais extensos.
Com uma maior extensão, aumentou-se a propensão à ocorrência de seções de duto não suportadas, chamadas de vãos-livres, devido às irregularidades do leito marinho, sejam elas preexistentes durante a instalação ou devido a subsequentes movimentos horizontais de \textit{scouring}\footnote{retirada de solo que suporta o duto devido às intensas correntes de fundo.} de dutos durante a operação.

A presença de trechos do duto em vão-livre exige uma avaliação para determinar a necessidade de ações corretivas para evitar danos aos mesmos.
Ainda na fase de projeto, uma avaliação do perfil do fundo do mar ao longo da rota proposta pode ser realizada para identificar se é esperado que haja trechos do duto em vão-livre.
Na existência de tais trechos, será necessária uma análise que forneça previsões dos números e tamanhos dos vãos esperados, que são indicadores da necessidade de possíveis alterações na rota ou ações corretivas.

Devido aos elevados custos (ambientais, financeiros, e à imagem da empresa), associados aos acidentes, o transporte seguro de hidrocarbonetos e outros fluidos nos oleodutos é uma das principais prioridades da indústria de petróleo e gás.
A vibração livre é uma grande preocupação na análise de fadiga de componentes de dutos submarinos, incluindo dutos em vãos-livres \cite{Gamino2013}.

Sendo assim, o comportamento estático e dinâmico do duto deve ser investigado para garantir a segurança, combatendo o dano estrutural por fadiga, mantendo-o em um estado aceitavelmente seguro.
Se as condições necessárias à segurança não puderem ser garantidas, as ações corretivas na forma de mudança de rota, correção de vãos, supressão do VIV e similares são usadas para garantir que os critérios de projeto relativos aos níveis de tensão e possíveis danos por fadiga devido ao VIV não sejam excedidos.
A análise requer o uso de métodos numéricos robustos para seu tratamento, e o Método dos Elementos Finitos (MEF) é amplamente usado nessa tarefa.
A configuração de dutos no fundo do mar depende das características topográficas do leito marinho, características do solo, tensão residual de lançamento, rigidez do duto e seu peso submerso.

Para que as condições de contorno e características do problema simulado reproduzam comportamento in loco, é necessário modelar desde a etapa de instalação até a operação do duto, assim como considerar efeito de carregamentos dos diferentes valores de pressões internas e externas nas respectivas etapas.
Modelar a instalação de dutos em um \textit{software} de elementos finitos para uso geral pode ser um trabalho demorado e tedioso, principalmente devido a grandes quantidades de dados da batimetria.
Na maioria das vezes, são necessárias técnicas avançadas de \textit{script} para definir o perfil do leito marinho e simular o processo de assentamento~\cite{VandenAbeele2013}.

Nesse cenário, são essenciais ferramentas que auxiliem no pré e pós-processamento de dados e na automação de procedimentos.
Uma ferramenta com essas características traz ganhos significativos para a produtividade e reduzem a probabilidade de erro humano.
Além disso, uma ferramenta que integre \textit{softwares} de uso específico (para análise e visualização, por exemplo), pode reduzir atritos do fluxo de trabalho, em comparação ao uso isolado destes \textit{softwares}.

Atualmente, existem diversos sistemas submarinos em operação nas Bacias de Campos e Espírito Santo que estão no final ou já ultrapassaram a metade de sua vida útil de projeto, o que torna ainda mais relevante uma ferramenta auxiliar para a reavaliação de integridade e extensão de vida operacional com critérios de cálculo validados.


\section{Cenário atual}


No cenário mundial existe a tendência da indústria de óleo e gás de investimento em transformação digital em todas as áreas da cadeia, com desenvolvimento de práticas e ferramentas. Esse movimento levou ao surgimento de ferramentas específicas ao auxílio do profissional responsável pela análise, visualização, predição dos resultados de VIV em dutos em vão-livre \cite{Mittal2017}. No entanto, pela especificidade dessas ferramentas, seu número ainda é reduzido, destacando apenas duas a nível comercial.


\subsection{SAGE Profile}


Desenvolvido pela empresa Fugro, que atual no monitoramento de dutos submarinos, o SAGE Profile \cite{sage_profile} é o software deles para análise de dutos submarinos. Por ser uma aplicação específica para este uso, esta aplicação representa avanços em relação a modelagem com um software de elementos finitos genéricos, a aplicação se limita a análise de elementos finitos, deixando a análise de fadiga a cargo do usuário. Além disso, a interação do usuário está limitada a interface gráfica (sem CLI\footnote{\textit{Command Line Interface}: interface de linha de comando}), o que dificulta a automação de tarefas corriqueiras.

\begin{figure}[!ht]
    \centering
    \caption{Interface gráfica do SAGE Profile.}\label{fig:sage_profile}
    \begin{subfigure}[t]{0.49\textwidth}
        \centering
        \includegraphics[width=\textwidth]{imagens/sage_profile_1}
    \end{subfigure}
    \hfill
    \begin{subfigure}[t]{0.49\textwidth}
        \centering
        \includegraphics[width=\textwidth]{imagens/sage_profile_2}
    \end{subfigure}
    \fonte{www.sage-profile.com.}
\end{figure}


\subsection{Sesam for pipelines}

A DNG-GL é uma referência mundial, entre outro áreas, em análise de dutos em vão-livre. Esta empresa é responsável pelo desenvolvimento da suíte \textit{Sesam for pipelines} \cite{dnvgl_sesam} focados na análise de dutos submarinos.
Essa súite consiste de 6 aplicações em VBA\footnote{Visual Basic for Applications, com uma interface de planilha do Microsoft Excel.}, dentre as quais a mais destacada é a \fatfree, responsável pelo cálculo da vida a fadiga em si. Sendo desenvolvidas pela DNG-GL, as aplicações seguem as recomendações práticas propostas pela mesma - o que trás bastante confiabilidade nos resultados. Entretanto apesar de conter uma aplicação para análise de comportamento mecânico, estas aplicação são simples, e estão muito aquém de um solução completa para simulação assentamento do tudo no solo, como o SAGE Profile.


% \section{Fluxo de avaliação de vida à fadiga em dutos em vão livre}\label{chap:workflow}


Baseado nos estudos e \textit{oficinas} realizados para o desenvolvimento deste trabalho. Pôde-se estabelecer que a análise de vida a fadiga em dutos em vão livre compreende o fluxograma apresentado na \autoref{fig:fluxograma}.

\begin{figure}[!ht]
    \centering
    \caption{Fluxo de avaliação de vida à fadiga em dutos em vão livre.}\label{fig:fluxograma}
    \includegraphics[width=\textwidth]{imagens/fluxograma.pdf}
    \fonte{Autor (2020)}
\end{figure}

A seguir, uma breve descrição de cada item:

\begin{enumerate}[label=(\arabic*)]
    \item Nesta etapa, o profissional reúne as informações básicas para construção dos modelos e outros dados usados em cálculos posteriores. Citadamente, temos aqui: os as cotas do perfil do duto e batimetria obtidas na inspeção, geometria e composição das camadas que compõem sessão do duto, parâmetros do solo, constantes físicas e coeficientes de segurança, posição e tipos de suportes ao longo do duto. Essa tarefa envole olhar uma série de documentos (\texttt{.doc}, \texttt{.pdf}, etc) em busca desses valores, dispostos de forma não estruturada. Quando estruturados, em forma de arquivos CSV ou planilhas, por exemplo, é necessário ainda manipular esses dados a fim de extrair somente a informação necessária ou convertê-las no formato apropriado. Um exemplo disso são os dados de batimetria, que precisam convertidos nas coordenadas dos nós de uma malha de elementos finitos.
    \item 
\end{enumerate}{}


\section{Objetivos}

Este trabalho tem como objetivo geral desenvolver uma ferramenta para a análise de fadiga em dutos submarinos em vãos-livres, que permita um fluxo de trabalho que inclua um software de análise de elementos finitos e uma planilha de cálculo de vida a fadiga.
Além disso, este trabalho tem como objetivos específicos:

\begin{itemize}
    \item Contribuir para a metodologia de análise de fadiga em dutos por meio da criação de uma metodologia de seleção de modos de vibração;
    \item Modelar e implementar a ferramenta utilizando o paradigma da programação orientada a objetos, através da linguagem \textit{Python};
    \item Criar um ambiente interativo de visualização de resultados da simulação com ferramentas que garantam maior liberdade e eficiência para o projetista;
    \item Validar a ferramenta com aplicação de casos.
\end{itemize}

% \section{Organização do Trabalho}

O \autoref{chap:viv}: apresenta os conceitos e as formulação por trás da Vibração Induzida por Vórtice (VIV), e as recomendações dados pela referência técnica \dnvf105. O \autoref{chap:assentamento}: apresenta a modelagem computacional do comportamento de duto submarinos, os passos de carga, e os tipos de elementos empregados para modelar os elementos necessários a simulação. O \autoref{chap:software}: apresenta os aspectos da implementação computacional do \textit{framework} , bem como os princípios norteadores de algumas escolhas da, a escolha da linguagem e paradigma de programação, a estrutura proposta para os módulos e classe.