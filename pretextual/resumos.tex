% ---
% RESUMOS
% ---

% RESUMO em português
\setlength{\absparsep}{18pt} % ajusta o espaçamento dos parágrafos do resumo
\begin{resumo}

    Com as descobertas de novos campos de petróleo em águas profundas, e distantes da costa, surgiu a necessidade de sistemas de transporte submarinos utilizando dutos rígidos cada vez mais extensos, e consequentemente com maior propensão à ocorrência de vãos-livres induzidos por irregularidades do piso marinho.
    A presença desses vãos traz a possibilidade de haver vibrações induzidas por vórtices (VIV) desprendidos pela passagem da corrente.
    Dessa forma é necessária uma avaliação que determina se há necessidade de ações corretivas para evitar danos aos mesmos.
    O início dessas e outras análises se baseiam no comportamento in-loco do duto.
    O Método dos Elementos Finitos é amplamente empregado na modelagem desse comportamento.
    No entanto, essa não é uma tarefa trivial uma vez que depende da utilização de vários softwares para etapas intermediárias.
    Além disso, essas análises envolvem manipulação de grandes quantidades de dados relacionados a batimetria do fundo marinho e propriedades de materiais, geometria do duto, correnteza, entre outros.
    Dessa forma, todo esse processo costuma ser demorado e passível de erros.
    Este trabalho apresenta o desenvolvimento de uma ferramenta para agilizar o fluxo de análise de fadiga em dutos submarinos sujeitos a vãos livre.
    

 \textbf{Palavras-chaves}: Dutos Submarinos; Modelagem Numérica; Modelagem Computacional.
\end{resumo}

% ABSTRACT in english
\begin{resumo}[Abstract]
    \begin{otherlanguage*}{english}
    
    With the discovery of new deep-sea oil fields far from the coast, has arisen the need for submarine transportation systems using increasingly large rigid pipelines, and consequently with greater propensity for spans induced by seabed irregularities. The presence of these spans brings the possibility of vortex-induced vibrations (VIV) produced by the current flow. Thus, an assessment is required to determine whether corrective actions are required to prevent damage to them. The start of these and other analyzes is based on in-situ pipeline behavior. The Finite Element Method is widely applied in modeling this behavior. However, this is not a trivial task as it depends on using various software for intermediate steps. Moreover, these analyses involve manipulation of large amounts of data related to seabed bathymetry and material properties, duct geometry, current, among others. This whole process is often time-consuming and error prone. This work presents the development of a tool to speed up the flow of fatigue analysis in subsea pipelines subjected to free spans.

    \vspace{\onelineskip}

    \noindent
    \textbf{Keywords}: Subsea Pipeline; Numerical Modeling; Computacional Modeling.
    \end{otherlanguage*}
\end{resumo}