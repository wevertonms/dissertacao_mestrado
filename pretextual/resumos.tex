% ---
% RESUMOS
% ---

% RESUMO em português
\setlength{\absparsep}{18pt} % ajusta o espaçamento dos parágrafos do resumo
\begin{resumo}
% TODO Um resumo deve conter os seguintes itens:
% 1) objetivo
% 2) contextualização/justificativa
% 3) metodologia para alcançar o objetivo
% 4) principais resultados alcançados ou esperados
% 5) principais contribuições
% Neste resumo, eu não vi os itens 3, 4 e 5.
% Além disso, o itens 2 foi explorado demasiadamente. Acho que ele pode ser mais condensado. Ao longo do texto essa justificativa pode ser mais explorada.

    A segurança no transporte de hidrocarbonetos e outros fluidos em dutos é um dos tópicos de maior atenção na indústria de petróleo e gás.
    Recentemente, essa indústria tem tido grandes preocupações sobre como o fenômeno da Vibração Induzida por Vórtice (VIV) afeta a vida à fadiga dos componentes e dutos submarinos, especialmente aqueles em vão-livre. % chktex 19
    Mundialmente, há várias dessas estruturas instaladas que já chegaram ao final ou metade de sua vida de projeto, o que torna imperativo revisitar as previsões iniciais e avaliar a necessidade de intervenção no sentido de estender a vida útil das mesmas, em especial no que diz respeito à VIV.
    Essas e outras análises se baseiam na correta representação no comportamento real do duto.
    O Método dos Elementos Finitos é amplamente empregado na modelagem desse comportamento.
    No entanto, essa não é uma tarefa trivial, uma vez que depende da utilização de vários softwares para etapas intermediárias.
    Além disso, essas análises envolvem manipulação de grandes quantidades de dados relacionados à batimetria do fundo marinho e propriedades dos materiais, geometria do duto, correnteza, entre outros.
    Dessa forma, todo esse processo costuma ser demorado e passível de erros.
    Concomitantemente, a indústria de óleo e gás tem passado por uma transformação digital nos anos recentes.
    Este cenário trouxe oportunidades para o desenvolvimento de novas ferramentas para o dia-a-dia do profissional responsável por estas análises.
    Neste contexto, este trabalho apresenta o desenvolvimento e estrutura de uma ferramenta computacional com funcionalidades do pré ao pós-processamento de dados da análise que visam melhorar o fluxo de análise de fadiga em dutos submarinos sujeitos a vãos-livres.
    % TODO: Comentar ao final um resumo dos resultados obtidos, positivos e ainda assim os que podem ser melhorados.

 \textbf{Palavras-chaves}: Dutos Submarinos; Análise de fadiga; Framework Computacional.
\end{resumo}

% ABSTRACT in english
\begin{resumo}[Abstract]
    \begin{otherlanguage*}{english}

    Safety in the transportation of hydrocarbons and other fluids in pipelines is one of the topics of greatest attention in the oil and gas industry.
    Recently, this industry has had great concerns about how the phenomenon of Vortex Induced Vibration (VIV) affects the fatigue life of submarine components and pipes, especially free-span pipelines.
    Worldwide, there are several of these structures installed that have already reached the end or half of their design lifespan, which makes it imperative to revisit the initial predictions and assess the need for intervention in order to extend their lifespan, especially with regard to to VIV.
    These and other analyzes are based on the correct representation of the pipeline's real behavior.
    The Finite Element Method is widely used to model this behavior.
    However, this is not a trivial task, since it depends on the use of various software for intermediate steps.
    In addition, these analyzes involve manipulation of large amounts of data related to the bathymetry of the seabed and material properties, pipe geometry, current, among others.
    Thus, this entire process is usually time-consuming and prone to errors.
    Simultaneously, the oil and gas industry has undergone a digital transformation in recent years.
    This scenario brought opportunities for the development of new tools for the professional responsible for these analyzes.
    This work presents the development and structure of a computational framework with functionalities from pre to post-processing of analysis data that aim to improve the flow of fatigue analysis in subsea pipelines subject to free spans.

    \vspace{\onelineskip}

    \noindent
    \textbf{Keywords}: Subsea Pipeline; Fatigue Analysis; Computational Framework Modeling.
    \end{otherlanguage*}
\end{resumo}