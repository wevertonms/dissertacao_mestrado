% ---
% RESUMOS
% ---

% RESUMO em português
\setlength{\absparsep}{18pt} % ajusta o espaçamento dos parágrafos do resumo
\begin{resumo}

    A segurança no transporte de hidrocarbonetos e outros fluidos em óleo é um dos tópicos de maior atenção na indústria de petróleo e gás.
    Recentemente, essa indústria tem tido grandes preocupações sobre como o fenômeno da Vibração Induzida por Vórtice (VIV) afeta a vida à fadiga dos componentes e dutos submarinos, especialmente os dutos em vãos-livre.
    Mundialmente, há várias dessas estruturas instaladas que já chegaram ao final ou metade de sua vida de projeto, o que torna imperativo revisitar as previsões iniciais e avaliar a necessidade de intervenção no sentido de estender a vida útil das mesmas, em especial no que diz respeito à VIV..
    O início destas e de outras análises se baseiam no comportamento in loco do duto.
    O Método dos Elementos Finitos é amplamente empregado na modelagem desse comportamento.
    No entanto, essa não é uma tarefa trivial, uma vez que depende da utilização de vários softwares para etapas intermediárias.
    Além disso, essas análises envolvem manipulação de grandes quantidades de dados relacionados à batimetria do fundo marinho e propriedades dos materiais, geometria do duto, correnteza, entre outros.
    Dessa forma, todo esse processo costuma ser demorado e passível de erros.
    Concomitantemente, a indústria de óleo e gás tem passado por uma transformação digital nos anos recentes.
    Este cenário trouxe novas oportunidades para o desenvolvimento de novas ferramentas para dia-a-dia do profissional responsável por estas análises.
    Este trabalho apresenta o desenvolvimento de um \textit{framework} computacional com funcionalidades do pré ao pós-processamento de dados da análise que visam melhorar o fluxo de análise de fadiga em dutos submarinos sujeitos a vãos-livres.



 \textbf{Palavras-chaves}: Dutos Submarinos; Modelagem Numérica; Modelagem Computacional; Fadiga.
\end{resumo}

% ABSTRACT in english
\begin{resumo}[Abstract]
    \begin{otherlanguage*}{english}

        Safe transportation of hydrocarbons and other oil fluids is one of the most important topics in the oil and gas industry.
        This industry has recently had major concerns about how the Vortex Induced Vibration (VIV) phenomenon affects the fatigue life of subsea components and pipelines, especially free span pipelines.
        Worldwide, there are several of these installed structures that have reached the end or half of their lifespan, which makes it imperative to revisit the initial forecasts and assess the need for intervention to extend their lifespan, especially with regard to VIV.
        The onset of these and other analyzes is based on in situ pipeline behavior.
        The Finite Element Method is widely employed in modeling this behavior.
        However, this is not a trivial task as it depends on using various softwares for intermediate steps.
        Moreover, these analyze involve manipulation of large amounts of data related to seabed bathymetry and material properties, pipeline geometry, current, among others.
        This whole process is often time-consuming and error prone.
        At the same time, the oil and gas industry has undergone a digital transformation in recent years.
        This scenario brought new opportunities for the development of new day-to-day tools of the professional responsible for these analyze.
        This paper presents the development of a tool to speed up the flow of fatigue analysis in subsea pipelines subjected to free spans.

    \vspace{\onelineskip}

    \noindent
    \textbf{Keywords}: Subsea Pipeline; Numerical Modeling; Computacional Modeling; Fatigue.
    \end{otherlanguage*}
\end{resumo}